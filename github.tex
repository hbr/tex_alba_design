\chapter{Github}





\section{Get Files}

Raw files can be downloaded from the content delivery network \emph{jsdelivr}.
\begin{verbatim}
    https://cdn.jsdelivr.net/gh/user/repo@version/filepath
\end{verbatim}
The version is optional. It can be a tag, a branch or a sha1.

Directly from github
\begin{verbatim}
    https://raw.githubusercontent.com/user/repo@version/filepath
\end{verbatim}






\section{Github API}

All API access is over https, and accessed from {\tt
  https://api.github.com}. All data is send and received as json.

\subsection{User Authentification}

In order to authenticate a user we invoke the command
{\small
\begin{verbatim}
    curl -u "username:pwd" https://api.github.com
\end{verbatim}%
}%
then the api responds either with
{\small
\begin{verbatim}
    {
      "message": "Bad credentials",
      "documentation_url": "https://developer.github.com/v3"
    }
\end{verbatim}
}
or a json with some information.

Request information about a repo and the tags of a repo
{\small
\begin{verbatim}
    curl -u "user:pwd" https://api.github.com/repos/user/repo
    curl -u "user:pwd" https://api.github.com/repos/user/repo/tags
\end{verbatim}
}

If a user wants to publish an Alba library we check the tags of the repo and
verify that the version is tagged in github. As a side effect we verify that
the user is really the owner of the repo.


Problem: During the execution of the curl command the password is visible via
{\tt ps -aef|grep curl}. The password might appear in the command history.



\section{Travis-ci}

Interesting travis environment variables:
\begin{description}

\item[TRAVIS\_COMMIT] The commit that the current build is testing.

\item[TRAVIS\_COMMIT\_RANGE] The range of commits that were included in the push
  or pull request. (Note that this is empty for builds triggered by the
  initial commit of a new branch.)
  
\item[TRAVIS\_OS\_NAME] On multi-OS builds, this value indicates the platform
  the job is running on. Values are currently linux, osx and windows (beta),
  to be extended in the future.

\item[TRAVIS\_REPO\_SLUG] The slug (in form: owner-name/repo-name) of the
  repository currently being built.


\item[TRAVIS\_PULL\_REQUEST\_SLUG] If the current job is a pull request, the slug
  (in the form owner-name/repo-name) of the repository from which the PR
  originated.

\end{description}
%
In order to check which files have been modified by a pull request we can
issue
{\small
\begin{verbatim}
   git diff --name-only $TRAVIS_COMMIT_RANGE
\end{verbatim}
}

We can use this to check that a pull request on the registry modifies only the
file \code{packages/user/repo/releases}.



\section{Automerge}

There is a tool called \code{mergify}. It is a github application which needs
to be activated on a github repository which receives the pull requests and
wants to merge the pullrequests automatically after some tests have been
passed.

The mergify service is free for open source projects. The application can be
installed either on all repos of a user or on selected repos of a
user. Question: What about organization users?


The configuration file is called \code{.mergify.yml}. It contains conditions
which must be satisfied to trigger a merge.

{\small
\begin{verbatim}
    pull_request_rules:
      - name: automatic merge on CI success and review
        conditions:
          - status-success=continuous-integration/travis-ci/pr
          - "#approved-reviews-by>=2"
        actions:
          merge:
            method: merge
\end{verbatim}
}

Note: A travis script can call git commands!


\section{Github Pages}

Content must be in the master branch, either rooted in the toplevel directory
or in the directory \code{docs} or in the gh-pages branch. The browser finds
the content at \code{user.github.io/repo}.

For user or organization sites the content is in the repo
\code{user.github.io}. The repo must be a repo of the user or the
organisation. Content must start at the root of the repo.


\paragraph{Automatic Deploy}
There is a basic script on
\url{https://github.com/domenic/zones/blob/master/deploy.sh} and a gist
\url{https://gist.github.com/domenic/ec8b0fc8ab45f39403dd}


Travis-ci has an own instruction on how to deploy to gh-pages at
\url{https://docs.travis-ci.com/user/deployment/pages/}.

On order to keep the source code branches and the gh-pages in the same repo, a
trick is required see
\url{http://onthecode.com/post/2015/06/14/master-and-gh-pages-branches-on-github.html}.

%%% Local Variables:
%%% mode: latex
%%% TeX-master: "main_alba_design"
%%% End:
