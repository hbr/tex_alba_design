\chapter{Elaboration}

The problem of elaboration is to read an Alba source code file and build terms
and definitions in the intermediate language. In order to do this, overloaded
symbols have to be resolved and implicit arguments have to derived.

During the development of Alba I initially have completely underestimated the
complexity of the task. In other dependently type languages like
Idris~\cite{brady2013}, Agda~\cite{norell2007} and Lean~\cite{demoura2015} papers
have been written to describe the complex procedure.

In Alba we have the additional complexity that Alba shall be language for
beginners and experts. The beginners should not be bothered with the complexity
of dependent types. Alba shall \emph{feel} like a simple functional programming
language. However the most powerful features of dependent types shall be
available in the language and the user should be able to grasp the power step by
step. I.e. the learning curve shall be flat and not steep.



\section{Implicit Arguments}
%%%%%%%%%%%%%%%%%%%%%%%%%%%%%





\section{Overloading}
%%%%%%%%%%%%%%%%%%%%%%%%%%%%%

Overloading means that the same name can be used for different
functions/constants. E.g. the operator $+$ can be addition for different type of
numbers like \code{Int} or \code{Float} or e.g. string concatenation. The
function \code{length} can be applied to a list, an array or a string argument.

Ambiguities shall be resolved in the context where the functions/constants are
used.

We have two different problems to solve:
\begin{enumerate}

\item Filter global ambiguous symbols before entering the context:
Functions/constants in the context with the same name shall be different either
for some explicit argument types or at the result type.

We shall not allow ambiguities in the global context which cannot be
distinguished by some explicit argument type or the result type.


\item Resolve ambiguities at usage: At the actual usage overloaded symbols shall
be resolvable.

\end{enumerate}

A type in normal form is either
\begin{enumerate}

\item a sort i.e. $\Prop$ or $\Any$.

\item a product $\Pi x^A. B$

\item a variable application $v a_0 a_1 \ldots$

\item a pattern match application
    $\case(\lambda \vec{y}^\vec{B} x^T. R) a_0 a_1 \ldots$
\end{enumerate}
%
Due to the typing rules neither lambda abstractions nor fixpoints can be types
(why not fixpoints???).





\subsection{Filter Overloaded Symbols}
%%%%%%%%%%%%%%%%%%%%%%%%%%%%%%%%%%%%%%

\subsubsection{First Try with Equivalence Relation}

In order to filter overloaded symbols we could try to find some equivalence
relation on types. Whenever two types are equivalent, there cannot be a name for
these two types. The equivalence relation shall satisfy the following
requirements:

\begin{itemize}

\item If two types are not equivalent, then both types must be distinguishable
by their explicit argument types and the result type.

\item The equivalence relation must not be too coarse to allow for useful
overloadings.

\end{itemize}

The optimal equivalence relation would be

\begin{quote}
    Two types are equivalent if they have the same number of explicit arguments
    and they cannot be distinguished by their argument types and the result
    type.
\end{quote}

The finest possible equivalence relation would be beta-equivalence i.e. two
types are considered equivalent if they have the same normal form. But this
equivalence relation is too fine. Consider the two types
%
\begin{alba}
    all {A: Any}: A -> A -> A

    Int -> Int -> Int
\end{alba}
%
which are not equivalent according to this equivalence relation.

However we cannot distinguish these two types by argument types and the result
type.

Given two arguments of type \code{Int} and a required result type of \code{Int},
both types are valid and an ambiguity cannot be resolved. Therefore we need a
more coarse equivalence relation which puts both types into the same equivalence
class.

A very coarse relation would be to consider two types equivalent which describe
a function with the same number of explicit arguments. But this relation is too
coarse, because it puts the following three types into the same equivalence class
%
\begin{alba}
    all {A: Any}: List A -> Natural

    all {A: Any}: Array A -> Natural

    String -> Natural
\end{alba}
%
With that equivalence relation we could not have a function named \code{length}
which works on lists, arrays and string.



\subsubsection{Preorder and Signatures}

If we look more into the details of the following types
%
\begin{alba}
    all {A: Any}: A -> A -> A

    Int -> Int -> Int

    String -> String -> String
\end{alba}
%
we see that an equivalence relation is not the appropriate tool. Certainly the
second and the third are not equivalent and could therefore be named with the
same symbol. The problem here: The first type is more general than the second
and the third. Therefore its to possible to have a symbol which has all three
types.

What we really need is some preorder on types. Let's call this preorder
\emph{includes}. Certainly \code{Int -> Int -> Int} includes itself. The types
\code{Int -> Int -> Int} and \code{String -> String -> String} do not figure in
the relation and \code{all \{A:Any\}: A -> A -> A} includes \code{Int -> Int ->
Int} and the corresponding string type as well.

The rule: We do not allow a symbol which has two possible types where one
includes the other.

In the following we transform a type into a signature and define a
\emph{includes} relation on the signatures. The \emph{includes} relation on
signatures is a partial order. Therefore the corresponding relation on types is
a preorder.

The primitive elements are global names (like \code{Int}, \code{List}), unknowns
\code{U} representing implicit arguments and \code{Sort}. Global names and sorts
do include only themselves. The unknowns include all other primitive elements.

Global names can be applied to arguments. We can form
\begin{alba}
    List U

    List Int

    Equal U U \end{alba}
%
The \emph{includes} relation extends to applications in a natural manner.
\code{List U} includes \code{List Int} and \code{List String}.

We use these elements to describe signatures. The operator $+$ has the
signatures
%
\begin{alba}
    (+): Int -> Int -> Int

        [Int, Int, Int]


    (+): String -> String -> String

        [String, String, String]
\end{alba}
%
The last type in a signature is always the final result type and the first types
are the signatures of the argument types.


Highly polymorphic functions have highly general signatures.
%
\begin{alba}
    (|>): all {A: Any} {B: A -> Any} (a: A) (f: all {x}: B x): B x

        [U, [U, U], U]
\end{alba}
%
Note that the description of the result type in this signature is unknown. It
can be anything even a function being able to take more arguments. Therefore the
number of actual explicit arguments of an application of a symbol of this
signature can be more than two.

The second explicit argument of this signature is a function type whose return
type is generic. If the function returns a function, it can be applied to more
arguments.

The most general type (i.e. the most inclusive type) is $\Pi A^\Any. A$ whose
signature is
\begin{alba}
    [U]
\end{alba}
%
This signature includes all other signatures. Therefore a symbol of type $\Pi
A^\Any. A$ does not allow any ambiguities. Fortunately a symbol of that type in
the global context would imply that the context is inconsistent, because
anything can be proved and generated having a symbol of this type.

But there are other signatures which are rather general. E.g. the signature of
the polymorphic identity function
%
\begin{alba}
    identity: all {A: Any}: A -> A

        [U, U]
\end{alba}
%
includes the signature of any other function with an arbitrary number of
arguments. Therefore it is practically impossible to overload the symbol
\code{identity}. Only local names can be \code{identity} because local names
shadow all other names introduced before.

All other functions which act like attributes can be easily overloaded. Let's
look into the example of the \code{length} function.

\begin{alba}
    length: all {A: Any}: List A -> Int

        [List U, Int]


    length: all {A: Any}: Array A -> Int

        [Array U, Int]
\end{alba}
%
However the introduction of a specific \code{length} function would cause an
illegal overloading
\begin{alba}
    length: all {A: Any}: List String -> Int

        [List String, Int]
\end{alba}
because the signature of the more general \code{length} function on lists above
includes this signature.


More example of signatures.
%
\begin{alba}

    Equal: all {A: Any}: A -> A -> Proposition

        [U, U, Sort]


    Plus: {A: Any} (R: Endorelation A): Endorelation A

        [[U,U,Sort], U, U, Sort]


    naturalInduction {P: Predicate Natural}
        (zero: P zero) (step: all {i}: P i -> P (add1 i))
        {n}: P n

        [U, [U,U], U]


    listInduction {A: Any} {P: Predicate (List A)}
        : P [] -> (all {a l}: P l -> P (a :: l)) -> {l} -> P l

        [U, [U,U], U]


    lessEqualInduction {R: Endorelation Natural}
        :   (all {n}: R zero n)
            -> (all {n m}: LessEqual n m -> R n m -> R (add1 n) (add1 m))
            -> all {n m}: LessEqual n m -> R n m

        [U, [LessEqual U U, U, U], LessEqual U U, U]

\end{alba}




\subsubsection{Computed Types}

In Alba types can be computed as well. Two examples to demonstrate a computed
type.
%
\begin{alba}
    (inspect n case
        zero := True
        add1 i := False)

    (inspect n case
        zero := Int
        add1 i := List Int)
\end{alba}
%
In computed types each branch of the pattern match can return a different type.
Therefore we have to allow set of types in signatures. The above two types have
the signatures
%
\begin{alba}
    {True, False}

    {Int, List Int}
\end{alba}

Note that computed types are always dependent. They depend on some object on
which the pattern match is done.





\subsection{Resolve Ambiguities}
%%%%%%%%%%%%%%%%%%%%%%%%%%%%%%%%

During the elaboration of a term, the possible ambiguities have to be resolved.
There are only two sources for ambiguity:
%
\begin{enumerate}

\item Overloaded global symbols like $+$ and \code{length}

\item Literal numbers (they can have type \code{Int}, \code{Float},
\code{Natural}, etc.).

\end{enumerate}




\section{Metavariables and Unification}
%%%%%%%%%%%%%%%%%%%%%%%%%%%%%%%%
