\section{Prelude}

The compiler makes sure that the following definitions are always
available. It is as if there were a module \code{alba.core.prelude} and each
module would start with
\begin{alba}
  use
     alba.core.prelude
     ...
\end{alba}



\subsection{Propositions}
\label{sec:prelude-propositions}


\subsubsection{Implication}
%
\begin{alba}
  (=>) (a b:Prop): Proposition
    := a -> b
\end{alba}

Propositions are types but unlike normal types we express propositions by
lowercase identifiers. The variables $a$ and $b$ are propositions, a term $p$
of type $a$ i.e. $p:a$ is a proof of the proposition.

The term $a\to b$ is just a shorthand for $\Pi x^a.b$ or
\begin{alba}
  all(x:a): b
\end{alba}
in program text. Therefore the term $0 < i \Rightarrow 0 \le i$ is a function
application with the value $\Pi p^{0 < i}. 0 \le i$.
\newline

\subsubsection{False}

The proposition \code{false} is a type which does not have any direct
proof. Therefore it is encoded as an inductive type with no constructor.
%
\begin{alba}
  class
    false: Proposition
  create
    -- no constructors
\end{alba}

The only possibility to prove \code{false} is to make some contradictory
assumptions. \vskip 2mm


\subsubsection{Negation}
%
\begin{alba}
  (not) (a:Proposition): Proposition
    := a => false
\end{alba}
\vskip 2mm


\subsubsection{Disjunction}
%
\begin{alba}
  class
    (or) (a b:Proposition): Proposition
  create
    left:  a => a or b
    right: b => a or b
\end{alba}

There are two ways to prove the expression \code{a or b} is to provide a proof
of the proposition $a$ and map it with the first constructor into a proof of
$a \lor b$ or provide a proof of $b$ and map it with the second constructor
into a proof of $a \lor b$.

The term \code{a => a or b} is a function application which evaluates to
\code{all (p:a) a or b} which is according to the typing rules a proposition
as well.
\newline


\subsubsection{Conjunction}
%
\begin{alba}
  class
    (and) (a b:Proposition): Proposition
  create
    conjunction: a => b => a and b
\end{alba}

In order to prove the proposition $a \land b$ we need a proof of $a$ and a
proof of $b$. The type of the constructor is a function type which maps a
proof of $a$ and a proof of $b$ to a proof of $a \land b$.
\newline


\subsubsection{Existential Quantification}

\begin{alba}
  class
     exist (A:Any) (f:A -> Proposition): Proposition
  create
     witness: all (a:A): f a -> exist A f
\end{alba}

\begin{itemize}
\item The type variable $A$ is implicit, because the type of the second
  argument depends on it. Therefore the compiler can infer it from the second
  argument.

\item $f:A\to \Prop$ means that $f$ is a predicate. It maps any element of
  type $A$ into a proposition (not into a proof of the proposition).

\item In order to construct a valid term of this inductive type we have to
  provide a witness $a$ and a proof showing that $f a$ (which is a
  proposition) is satisfied.
\end{itemize}

Example
\begin{alba}
  exist (\(i:Natural) := 0 < i)
\end{alba}


\subsection{Strong types}
\label{sec:prelude-strong-types}

Strong types are types with a computational and a proposition content. At
runtime all propositional content is erased. The propositional context serves
the compiler to carry information of proofs of certain properties which it can
use to proof more properties.


\subsubsection{Decision}
%
\begin{alba}
  class
    Decision (a b:Proposition): Any
  create
    left:  a -> Decision a b
    right: b -> Decision a b
\end{alba}

A term representing a decision between two propostions $a$ and $b$ needs
either a proof of $a$ or a proof of $b$. At runtime the proofs are erased and
only the information about the used constructor remains. Therefore a decision
value is like an enriched boolean value where the enrichment is removed at
runtime.


\subsubsection{Specified type}

\begin{alba}
  class
     Specified (A:Any) (f:A -> Proposition): Any
  create
     specified: all (a:A) f a -> Specified A f
\end{alba}

In the intermediate language this inductive definition is written as
$$
\Inductive_{[A:\Any, f:A \to \Prop], I:\Any, [\Pi a^A . f a \to I]}.
$$





%%% Local Variables:
%%% mode: latex
%%% TeX-master: "main_alba_design"
%%% End:
