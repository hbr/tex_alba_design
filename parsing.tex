\chapter{Parsing}

\section{Incremental Parsing with Combinators}

\subsection{Basics of an Incremental Parser}

An incremental parser receives one token at a time. At any point it gives
information whether it needs more tokens or it has terminated.

In order to work properly the parser has to buffer some information.
\begin{itemize}

\item In order to process alternatives it has to be able to store at least one
  lookahead token. If it receives one token it tries the first alternative. If
  the alternative fails on the first token, the parser has to put the token
  into the lookahead buffer and try the next alternative.

\item Sometimes an alternative fails after having consumed some tokens. In
  order to process the next alternative the consumed tokens of the first
  failed alternative have to be stored into some consumption buffer. After
  failure the tokens of the consumption buffer have to be pushed back onto the
  lookahead buffer and the next alternative can be tried. This processing is
  possible only if the first alternative is processed as backtrackable
  alternative.
\end{itemize}

Backtracking is costly, therefore it should be avoided if possible. However
some language constructs can be parsed only with backtracking information.

An incremental parser has to be able to store consumption information and
lookahead information beside other information (see below). The parser sees a
certain window of the input stream.

$$
\underbrace{c_0\,c_1, \ldots}_{\text{consumed}}
\quad
\underbrace{b_0\,b_1, \ldots}_{\text{buffered}}
\quad
\underbrace{l_0\,l_1, \ldots}_{\text{lookahead}}
\quad
\underbrace{u_0\,u_1, \ldots}_{\text{unseen}}
$$
%
Only the buffered (i.e. consumed and buffered) and the lookahead information
is stored.


\subsection{Interface}

The following specifies the needed types and function to operate the parser.
\begin{ocaml}
  type state           (* to store e.g. line and column numbers *)
  type token           (* Consumed one at a time or rejected *)
  type final           (* Final result of the parser *)
  type error           (* Possible dead end *)

  type parser

  val put_token:  parser -> token -> parser  (* Process one token *)

  val needs_more: parser -> bool
  val has_ended:  parser -> bool

  val state:  parser -> state

  val result: parser -> (final,error list) result (* require: has_ended *)
  val lookahead: parser -> token list             (* require: has_ended *)
\end{ocaml}

At the end of parsing we can ask for the final result of the parser. The
parser either returns an object of type \verb!final! or a list of paths it has
tried which ended in an error i.e. a list of dead ends.

There are intermediate parsers of type \code{'a t} which either succeed with a
value of type \code{'a} or produce a list of dead ends. There are 3 functions
to produce elementary intermediate parsers
%
\begin{ocaml}
  val succeed: 'a -> 'a t            (* Immediately succeed *)
  val fail:    error -> 'a t         (* Immediately fail    *)
  val token:   (state -> token -> ('a*state,error) result) -> 'a t
            (* Look at a token and either consume it and succeed or fail *)
\end{ocaml}
%
and further functions to combine intermediate parser to more complex parsers
%
\begin{ocaml}
  val (>>=): 'a t -> ('a -> 'b t) -> 'b t  (* and then     *)
  val (<|>): 'a t -> 'a t -> 'a t          (* alternatives *)
  val (<?>): 'a t -> error -> 'a t         (* replace error messages *)
  val backtrackable: 'a t -> 'a t          (* buffer consumed tokens for backtracking *)
  val commit: 'a -> 'a t                   (* commit on next consumption,
                                              only within backtrackable *)
\end{ocaml}
%
and a function to produce a parser from an intermediate parser of the final
type
%
\begin{ocaml}
  val make_parser: state -> final t -> parser
  (* Produce a parser which starts at some initial state and is able to
     return some final result. *)
\end{ocaml}


\subsection{Implementation}

\subsubsection{Final Parser}
The final parser has two states, either it has ended or it needs more
tokens. We represent a parser as an algebraic type and implement some
functions
%
\begin{ocaml}
  type parser =
  | More  of state * (state -> token -> parser)
  | Final of state * (final, error list) result * token list

  let needs_more (p:parser): bool =
    match p with
    | More _ -> true | Final _ -> false

  let has_ended (p:parser): bool = not (needs_more p)

  let put_token (p:parser) (t:token): parser =
    match p with
    | More (st,f) ->
        f st t
    | _ ->
      assert false (* Illegal call *)

  let state (p:parser): state =
    match p with
    | More (st,_) | Final (st,_,_) -> st

  let result (p:parser): (final,error list) result =
    match p with
    | Final (_,r,_) -> r
    | _ -> assert false (* Illegal call! *)

  let lookahead (p:parser): token list =
    match p with
    | Final (_,_,la) -> la
    | _ -> assert false (* Illegal call! *)
\end{ocaml}
%


\subsubsection{Intermediate Parsers}
An intermediate parser knows just how to produce an intermediate value or a
list of dead ends. It is not yet a complete parser. It is a function with two
arguments: a continuation which knows how to parse the remaining input and
some buffer information
%
\begin{ocaml}
  type 'a t = ('a option -> parser) -> B.t -> parser
\end{ocaml}
%

During the implementation of the parser we derive step by step the
requirements for the buffer module \code{B}. Since the buffer is not visible
outside the parser we can implement it as a mutable structure.

The implementation of the function \code{parser} looks like
\begin{ocaml}
  let make_parser (s:state) (p:final t): parser =
    let b = B.make s in
    p (fun o ->  (* continuation function for p *)
         Final (B.state b,
                (match o with
                 | Some x -> Ok x
                 | None   -> Error (B.errors b)),
                B.lookahead b))
      b
\end{ocaml}
%
Note that the buffer is a mutable structure. During the execution of \code{p}
it will be mutated and the continuation function will be called with the final
value of the buffer.


\subsubsection{Elementary Parsers}
The elementary parsers \code{succeed} and \code{fail} are
straightforward.
\begin{ocaml}
  let succeed (a:'a) (k:'a option -> parser) (b:B.t): parser =
    k (Some a)

  let fail (e:error) (k:'a option -> parser) (b:B.t): parser =
    B.add_error e b;
    k None
\end{ocaml}
%
Note that the function \verb!fail! modifies the buffer \emph{before} calling
the continuation. Furthermore note that the type of \code{succeed} is \code{'a
  -> 'a t} and the type of \code{fail} is \code{error -> 'a t} because the
type \code{'a t} represents the type of the remaining function.

It is instructive to see which final parsers can be produced from
\code{succeed} and \code{fail} assuming that \code{'a} represents the final
type. The function call \code{make\_parser s (succeed v)} returns
%
\begin{ocaml}
  Final (s, Ok v, [])
\end{ocaml}
%
and the function call \code{make\_parser s (fail e)} returns
%
\begin{ocaml}
  Final (s, Error [e], [])
\end{ocaml}
%
This fact can be verified by expanding the definitions of \code{succeed},
\code{fail} and \code{make\_parser}.

Both parsers do not process any tokens, they immediately succeed or fail with
the initial state and without leaving a lookahead.

The third elementary parser which processes one token is not complicated
either.
\begin{ocaml}
  let token (f:state -> token -> ('a*state, error) result)
            (k:'a option -> parser)
            (b:B.t): parser =
    More (B.state b,
          fun s0 t ->
            match f s0 t with
            | Ok (a, s1) ->
                B.consume t s1 b;
                k (Some a)
            | Error e ->
                B.reject t e b;
                k None)
\end{ocaml}

Let's see what parser is produced by \code{make\_parser s0 (token
  f)}. Regardless of \code{f} the returned parser is \code{More (...)} which
can receive tokens and is not yet in its final state.

Therefore it is better to ask what is produced by
\begin{ocaml}
  put_token (make_parser s0 (token f)) t
\end{ocaml}
Now it depends on the result of \code{f s0 t}. If the result is \code{Ok
  (a,s1)} then the result of the above expression is
%
\begin{ocaml}
  Final (s1, Ok a, [])
\end{ocaml}
%
i.e. the token is consumed (therefore no lookahead) and the state \code{s1}
and the value \code{a} is returned.

If the result of \code{f s0 t} is \code{Error e} then the result of the above
expression is
%
\begin{ocaml}
  Final (s0, Error [e], [t])
\end{ocaml}
%
i.e. the token is not consumed (therefore the lookahead), the initial state
and the list of dead ends consisting of one item is returned.

The intermediate parser \code{token f} either consumes a token and returns a
result or it rejects a token, puts it onto the lookahead buffer and adds a
dead end to the buffer.

\subsubsection{Building Parsers from Simpler Parsers}

It is not very interesting to produce parsers which either succeed or fail
immediately or process one token and then succeed or fail. In order to produce
more interesting parsers we have be able to combine parsers and build step by
step from the elementary parsers the final parser.

The sequencing combinator \code{>>=} and the alternative combinator \code{<|>}
(see below) execute the first parser and in case of success or error without
consumption execute the second parser. Since the first parser might leave
lookahead tokens in the lookahead buffer, we need a function to apply the
lookaheads to the second parser.

\begin{ocaml}
  let apply_lookahead (b:B.t) (p:parser): parser =
    if B.has_lookahead b && needs_more p  then
      let t = B.pop_one_lookahead b in
      put_token p t
    else
      p
\end{ocaml}
%
If the parser needs more tokens, then the next elemenary parser must be the
\code{token} parser. Reason: No other elementary parser produces a \code{More
  ...}. The \code{put\_token} either consumes the token or puts it back into
the lookahead buffer.

Both combinators \code{>>=} and \code{<|>} call \code{apply\_lookahead} before
executing the second parser. Therefore as long as there is an elementary
\code{token} parser, lookahead tokens are extracted from the buffer and
applied to the parser. As soon as there is no more elementary \code{token}
parser, the parsing has ended in a final state.



\subsubsection{Sequencing \code{>>=}}

If we have an intermediate parser \code{p:'a t} it either produces a result of
type \code{'a} or a list of dead ends. In sequencing we apply the first parser
and in case of success we can use the value to compute the next parser or in
case of failure the whole sequence fails. Therefore the sequencing operator
\code{>>=} has the signature
%
\begin{ocaml}
  val (>>=): 'a t -> ('a -> 'b t) -> 'b t
\end{ocaml}
%
The implementation is straightforward
%
\begin{ocaml}
  let (>>=) (p:'a t) (f:'a -> 'b t) (k:'b option -> parser) (b:B.t): parser =
     p (fun o ->  (* continuation function for p *)
          match o with
          | Some a ->  (* p might have left over some lookahead *)
            apply_lookahead b (f a k b)
          | None ->
            k None)
       b
\end{ocaml}



\subsubsection{Alternatives \code{<|>}}

If we have two intermediate parsers \code{p} and \code{q} of type \code{'a t}
then we first execute \code{p} and if successful we take that result as the
result of the combination \code{p <|> q}. If \code{p} fails without consuming
tokens then we execute \code{q} and take this as a result of the combination.

In order to see whether \code{p} has consumed tokens we have to reset the
\emph{consumed} flag in the buffer before executing \code{p} and restore it
afterwards
%
\begin{ocaml}
  let (<|>) (p:'a t) (q:'a t) (k:'a option -> parser) (b:B.t): parser =
     assert (B.no_lookahead b);
     let c0 = B.reset_consumed b in
     p (fun o ->
          let c1 = B.has_consumed b in
          B.set_consumed (c0 || c1) b;
          match o with
          | None when not c1 -> (* p did not consume tokens *)
            apply_lookahead b (q k b)
          | _ ->
            (* p either succeeded or failed and consumed tokens *)
            k o)
       b
\end{ocaml}





\subsubsection{Replace Error Messages \code{<?>}}

When we execute several alternatives \code{p <|> q <|> ...} and all of them
fail, each failure leaves some dead end in the buffer structure.

Sometimes we want to replace the list of error messages with a more high level
error message. In order to do this we have the operator \code{<?>)} with the
signature
%
\begin{ocaml}
  val (<?>): 'a t -> error list -> 'a t
\end{ocaml}
%
and the implementation
%
\begin{ocaml}
  let (<?>) (p:'a t) (es:error list) (k:'a option -> parser) (b:B.t): parser =
    assert (B.no_lookahead b);
    let e0 = B.reset_errors b in
    p (fun o ->
          (match o with
           | None -> B.set_errors es b
           | _ ->    ());
          B.update_errors e0 b;
          k o)
      b
\end{ocaml}
%



\subsubsection{Backtracking}

Sometimes we want to make a parser \emph{backtrackable} i.e. in case of
failure where some tokens have already been consumed we want to undo the
consumption (put them back onto the lookaheads) and pretend that nothing has
been consumed. Remember other alternatives of a parser can only be tried if
the preceeding parser has failed without consumption.


The implementation of the function \code{backtrackable:'a t -> 'a t} is
straightforward because we push the complexity of backtracking into the buffer
structure.
%
\begin{ocaml}
  let backtrackable (p:'a t) (k:'a option -> parser) (b:B.t): parser =
    let back = B.start_backtrack b in
    p (fun o ->
         (match o with
          | Some _ -> B.end_backtrack_success back b
          | None ->   B.end_backtrack_fail    back b);
         k o)
      b
\end{ocaml}


Sometimes we do not want to make an intermediate parser completely
backtrackable but only to a certain point. The function \code{commit} marks
this point. If a backtrackable parser consumes tokens after a \emph{commit}
then it is no longer backtrackable i.e. that specific path is committed an no
other alternatives can be explored any more.
%
\begin{ocaml}
  let commit (a:'a) (k:'a option -> parser) (b:B.t): parser =
    B.commit b;
    k (Some a)
\end{ocaml}
%
Having this function we can program a pattern
%
\begin{ocaml}
  backtrackable (p >>= commit >>= f)
\end{ocaml}
%
The expression \code{p >>= commit >>= f} works like \code{p >>= f} except that
after succeesful termination of \code{p} consumed tokens are no more buffered
and the whole parser is no more backtrackable. Only failures within \code{p}
are backtrackable.







\subsection{The Buffer of the Parser}


\subsubsection{Types}

\begin{ocaml}
  (* The buffer module, i.e. module B *)
  type consumed     = bool
  type is_buffering = bool
  type consumption_length = int

  type commit = Not | Committing | Committed
  type back = state * consumed * consumption_length
              * error list
              * is_buffering * is_buffering  * commit

  type saved_errors = consumed * error list

  type t = {mutable state: state;
            mutable consumed:      consumed;
            mutable consumption:   token list;
            mutable n_consumption: consumption_length;
            mutable lookahead:     token list;
            mutable errors:        error list;
            mutable isbuf:         is_buffering;
            mutable isbuf_prev:    is_buffering;
            mutable commit:        commit;
           }
\end{ocaml}



\subsubsection{Query Functions}

\begin{ocaml}
  let no_lookahead (b:t): bool = b.lookahead = []
  let lookahead    (b:t): token list = List.rev b.lookahead
  let errors       (b:t): error list = List.rev b.errors
  let has_consumed (b:t): bool = b.consumed
  let state        (b:t): state = b.state
\end{ocaml}


\subsubsection{Initialization etc.}

\begin{ocaml}
  let init (s:state): t =
    {state = s;
     consumed = false; consumption = []; n_consumption = 0;
     lookahead = [];
     errors = [];
     isbuf = false; isbuf_prev = false; commit = Not}

  let pop_one_lookahead (b:t): token =
    match b.lookahead with
    | [] -> assert false (* Illegal call *)
    | t :: rest ->
       b.lookahead <- rest;
       t

  let add_error (e:error) (b:t): unit =
    b.errors <- e :: b.errors
\end{ocaml}

\subsubsection{Support for Parser 'token'}

\begin{ocaml}
  let consume (t:token) (s:state) (b:t): unit =
    assert (no_lookahead b);
    b.consumed <- true;
    b.state    <- s;
    b.errors   <- [];
    if b.isbuf then                                    (* 1 *)
       (b.consumption   <- t :: b.consumption;         (* 2 *)
        b.n_consumption <- b.n_consumption + 1;
        if b.commit = Committing then
          (b.commit <- Committed;                      (* 3 *)
           if not b.isbuf_prev then                    (* 4 *)
             b.isbuf <- false))
\end{ocaml}

\begin{enumerate}
\item A parser is buffering if it is part of a backtrackable parser which has
  not yet committed.
\item Put the consumed token into the consumption buffer.
\item If a commit has been issued (i.e. \code{Committing}), the next consumed
  token sets the \code{commit} to \code{Committed}.
\item A \emph{committed} parser does no longer buffer consumed tokens unless
  an outer parser is \emph{buffering}.
\end{enumerate}

\begin{ocaml}
  let reject (t:token) (e:error) (b:t): unit =
    assert (no_lookahead b);
    b.errors <- e :: b.errors;
    b.lookahead <- [t]
\end{ocaml}


\subsubsection{Support for Combinator \code{<|>}}

\begin{ocaml}
  let reset_consumed (b:t): bool =
     let c = b.consumed in
     b.consumed <- false; c
  let set_consumed (c:bool) (b:t): unit =
     b.consumed <- c
\end{ocaml}

\subsubsection{Support for Combinator \code{<?>}}

\begin{ocaml}
  let reset_errors (b:t): saved_errors =
     let saved = b.consumed, b.errors in
     b.consumed <- false;  (* to be able to check if the next parser has
                              consumed anything *)
     b.errors <- [];       (* in order to collect the error messages generated
                              by the next parser *)
     saved

  let set_errors (errs:error list) (b:t): unit =
     b.errors <- errs

  let update_errors ((c,errs):saved_errors) (b:t): unit =
     if b.consumed then
       () (* parser has advanced, old errors are obsolete *)
     else
       (b.consumed <- c;             (* restore consumption flag *)
        b.errors <- b.errors @ errs) (* parser has not consumed, old error
                                        messages are still valid *)
\end{ocaml}
%

\subsubsection{Support for \code{backtrackable} and  \code{commit}}

\begin{ocaml}
  let move_buffered (n:int) (la:bool) (b:t): unit =
    (* Remove the buffered consumed tokens so that the consumption buffer
       has only n tokens and put them into the lookahead buffer if the la
       flag is set. *)
    assert (n <= b.n_consumption);
    while n < b.n_consumption do
      b.n_consumption <- b.n_consumption - 1;
      match b.consumption with
      | [] -> assert false (* cannot happen *)
      | t :: rest ->
          b.consumption   <- rest;
          b.n_consumption <- b.n_consumption - 1;
          if la then
            b.lookahead <- t :: b.lookahead
    done

  let start_backtrack (b:t): back =
    let saved = b.state, b.consumed, b.n_consumption,
                b.errors, b.isbuf, b.isbuf_prev, b.commit in
    b.consumed <- false;
    b.isbuf_prev <- b.isbuf;
    b.isbuf <- true;
    b.errors <- [];
    b.commit <- Not;
    saved

  let end_backtrack_success ((_,c,n,es,ib,ibp,comm):back) (b:t): unit =
    if not ib then
      (* no buffering previously, therefore remove all newly buffered tokens. *)
      move_buffered n false b;
    if not b.consumed then
      (* backtrackable parser has not consumed tokens, errors must be restored *)
      b.errors <- es;
    b.consumed   <- c && b.consumed;
    b.isbuf      <- ib;
    b.isbuf_prev <- ibp;
    b.commit     <- comm

  let end_backtrack_fail    ((s,c,n,es,ib,ibp,comm):back) (b:t): unit =
    if b.commit <> Committed then (* undo consumption *)
      (move_buffered n true b;
       b.consumed <- c;
       b.state <- s;
       b.errors <- b.errors @ es)
    else
       b.consumed   <- c && b.consumed;
    b.isbuf      <- ib;
    b.isbuf_prev <- ibp;
    b.commit     <- comm

  let commit (b:t): unit =
    if b.isbuf && b.commit = Not then
      b.commit <- Committing
\end{ocaml}






\section{Command Line Parsing}



\section{Parsing Problems in Alba}

\subsection{Newline as Optional Separator}

A predictive parser cannot work in parsing Alba sourcecode. Reason: Whitespace
with newline might be significant as a separator.

\begin{alba}
  id(A:Any, a:A): A :=
    a        -- nl is separator

  add(i,j:Nat): Nat :=
    i        -- nl is ignored
    + j
\end{alba}
%
In the first declaration the newline after the expression \verb!a! is a
separator and in the second declaration the newline after the operand \verb!i!
has to be ignored, because the expression continues.

An LL(1) parser cannot decide on seeing the newline whether to terminate the
declaration and looking for a new declaration or to look for further operands.

We can use the following grammar to describe a sequence of declarations.
{\scriptsize
\begin{verbatim}
    decls   ::= eps | decl declseq
    declseq ::= eps | sep decl declseq
    sep     ::= nl  | ';'
    decl    ::= hd nlopt exp
    hd      ::= id ... ':='
    nlopt   ::= eps | nl
    exp     ::= id ops
    ops     ::= eps | nlopt op nlopt id ops
\end{verbatim}
}
%
In order to compute the parsing table for a recursive descent parser we need
the first and follow set for each nonterminal of the grammar.
{\scriptsize
\begin{verbatim}
    nonterminal       first            follow

    decls             {eps,id}         {$}
    declseq           {eps, nl, ';'}   {$}
    sep               {nl, ';'}        {id}
    hd                {id}             {nl,id}
    nlopt             {eps,nl}         {op, id}
    exp               {id}             {$, nl, ';'}
    ops               {eps,nl,op}      {$, nl, ';'}
\end{verbatim}
}
%
Parsing table
{\scriptsize
\begin{verbatim}
    entry(decls)   = {id -> decls ::= decl declseq,
                      $  -> decls ::= eps}
    entry(declseq) = {(nl,';') -> declseq ::= sep decl declseq,
                      $        -> declseq ::= eps}
    entry(sep)     = {nl -> sep ::= nl, ';' -> sep ::= ';'}
    entry(decl)    = {id -> decl ::= hd nlopt exp}
    entry(hd)      = {id -> hd ::= id ... ':='}
    entry(nlopt)   = {nl -> nlopt ::= nl, (op,id) -> nlopt ::= eps}
    entry(exp)     = {id -> exp ::= id ops}
    entry(ops)     = {(nl,op) -> ops ::= nlopt op nlopt id ops,
                      ($,nl,';') -> ops ::= eps}
\end{verbatim}
}
%
The nonterminal \verb!ops! has two entries for the terminal \verb!nl! i.e. a
predictive parser cannot decide which of the two productions to use when
trying to parse \verb!ops! on seeing \verb!nl!.

The only solution for a recursive descent parser is to backtrack i.e. first
use one production and backtrack as soon as an error occurs and try then the
other production.

Note: Backtracking is expensive because all tokens need to be bufferend until
the error occurs and then replayed onto the alternative. Therefore a tokenizer
should put as much information into the significant whitespace tokens in order
to minimize the backtracking information.


\subsection{Documentation Comments}

The comments serving as documentation are expected.

\begin{alba}
  add(i,j:Nat): Nat :=
      -- The sum of the two number i and j
    i + {: some comment :} j
\end{alba}

Convention: An expected comment has at least one newline (all comments
starting with \verb!--! span to the end of line, therefore they have one
newline). Only nested comments might not have newlines. Nested comments
without newlines are parsed as whitespace and completely ignored by the
tokenizer. Comments with newline are returned as an newline token by the
tokenizer.





%%% Local Variables:
%%% mode: latex
%%% TeX-master: "main_alba_design"
%%% End:
