
\chapter{Certified Programming}
\label{chap:certprog}





\newpage
\section{Logic}
\label{sec:certprog-logic}


\subsection{Negation}

Negation is defined in the prelude as
\begin{alba}
  (not) (a: Proposition): Proposition :=
    a => false
\end{alba}

We want to prove the contrapositive law of logic which states
\begin{alba}
  (a => b) => not b => not a
\end{alba}
for all propositions \code{a} and \code{b}. A proof of that fact would be
written in Alba as the function
%
\begin{alba}
  contrapositive a b (ab: a => b) (nb:not b): not a :=
    _      -- compiler, please generate a proof of 'not a'!
\end{alba}
%
It is not necessary to write such a proof by hand, because the compiler can
generate such a proof automatically. But if you want to write the proof
explicitly it looks like

\begin{alba}
  contrapositive a b (ab: a => b) (nb: not b): not a :=
    \ pa :=
        explicit_arguments
          nb (ab pa)
\end{alba}

\noindent Explanation: A proof of \code{not a} is a function which maps a
proof of \code{a} into a proof of false. If we have a proof of \code{a} we can
use the argument \code{ab} which maps a proof of \code{a} into proof of
\code{b} and then use \code{nb} which maps a proof of \code{b} into a proof of
\code{false} which is the desired result.

Since proof arguments are implicit in Alba, we have to use the keyword
\code{explicit\_argmuments} to give them explicitly.




\subsection{Disjunction}

Logical dijunction is defined in the prelude as the inductive type
%
\begin{alba}
  class
    (or) (a b: Proposition): Proposition
  create
    left:  a => a or b
    right: b => a or b
\end{alba}


\begin{alba}
  swap_or a b (p:a or b): b or a :=
    explicit_arguments
      inspect
        p
      case
        left  pa := right pa
        right pb := left  pb
\end{alba}

\begin{alba}
  eliminate_or a b c (ab: a or b) (ac: a => c) (bc: b => c): c :=
    explicit_arguments
      inspect
        ab
      case
        left  pa :=
          ac pa
        right pb :=
          bc pb
\end{alba}


\subsection{Conjunction}

\begin{alba}
  class
    (and) (a b: Proposition): Proposition
  create
    conjunction: a => b => a and b
\end{alba}


\begin{alba}
  swap_and (a b: Proposition) (p:a and b): b and a :=
    explicit_arguments
      inspect
        p
      case
        conjunction _ _ pa pb := conjunction b a pb pa
\end{alba}


\begin{alba}
  eliminate1_and (a b: Propositon) (p: a and b): a :=
    inpect p case
      conjunction _ _ pa pb := pa
\end{alba}







\newpage
\section{Relations}
\label{sec:certprog-relations}


\subsection{Basic Properties}


In the following we consider endorelations i.e. binary relations where both
domains are the same. For endorelations we can define what it means to be
reflexive, transitive and symmetric.

\begin{alba}
  is_reflexive (A:Any) (r: Endorelation A): Proposition :=
    all a: r a a

  is_transitive (A:Any) (r: Endorelation A): Proposition :=
    all a b c: r a b => r b c => r a c

  is_symmetric (A:Any) (r: Endorelation A): Proposition :=
    all a b: r a b => r b a
\end{alba}


Next we define the transitive closure of a relation.

\begin{alba}
  class
    plus (A:Any) (r: Endorelation A): Endorelation A
      -- 'plus r' is the transitive closure of 'r'
  create
    init a b:
      r a b
      => plus r a b
    step a b c:
      r a b
      => plus r b c
      => plus r a c
\end{alba}

The definition uses an inductive type. It uses two constructors. The
\code{init} constructor allows us to conclude \code{plus r a b} from \code{r a
  b}.
The \code{step} constructor allows us to conclude \code{plus r a c} from
\code{r a b} and \code{plus r b c}.

By intuition we see that the relation \code{plus r} is transitive. But in
order to be sure we need a proof. In order to proof transitivity we have to
prove
%
\begin{alba}
  plus r a b => plus r b c => plus a c
\end{alba}
%
i.e. we need a function which looks like
%
\begin{alba}
  f a b c (pab: plus r a b) (pbc: plus r b c): plus r a c :=
    ...
\end{alba}
%
In order to this we can do an induction proof on \code{pab: plus r a b} which
generates two cases.

One case is that \code{plus r a b} has been constructed
by \code{init a b rab}. In that case we can use the step constructor with
\code{ab: r a b} and \code{pbc: plus r b c} to construct \code{plus r a c}.

The second case is that  \code{plus r a b} has been constructed by \code{step
  a x b (ax: r a x) (pxb: plus r x b)}. Now we can use the induction
hypothesis to construct from \code{pxb} and \code{pbc} a proof of \code{plus r
  x c} and then the step constructor to generate a proof of \code{plus r a c}.


\begin{alba}
  plus_transitive A (r:Endorelation A): (plus r).is_transitive :=
    f a b c (pab: plus r a b) (pbc: plus r b c): plus r a c :=
      inspect
        pab
      case
        init a b :=   -- implicit argument '_: r a b'
            -- a -> b +> c
          step a b c

        step a x b :=
            -- implicit arguments _:  r a x
            --                    _: plus r x b
            {: goal: a -> x +> b +> c
               by a recursive call to f we prove x +> c
               and then use the step constructor to prove a +> c :}
          step a x c
          where
             _: plus r x c := f x b c
\end{alba}



In the same manner as the transitive closure we can define the reflexive
transitive closure of a relation.

\begin{alba}
  class
    star (A:Any) (r: Endorelation A): Endorelation A
      -- 'star r' is the reflexive transitive closure of 'r'
  create
    init a:
      star r a a

    step a b c:
      r a b
      => star r b c
      => star r a c
\end{alba}

Having the definition we prove that the reflexive transitive closure is
transitive.

\begin{alba}
  star_transitive A (r:Endorelation A): (star r).is_transitive :=
    f a b c (sab: star r a b) (sbc: star r b c): star r a c :=
      inspect
        sab
      case
        init a :=
          sbc
            -- a = b in this case

        step a x b :=
          step a x c
            -- using induction hypo: f x b c: star r x c
\end{alba}


\subsection{Diamonds and Confluence}


A relation is a diamond if it is always possible to join two steps with a
single step.

\begin{alba}
  is_diamond A (r: Endorelation A): Proposition :=
      {:   a  ->  b
           |      |
           v      v
           c  -> some d :}
    all a b c:
      r a b
      => r a c
      => some d: r b d and r c d
\end{alba}


A relation is confluent if its reflexive transitive closure is a diamond.

\begin{alba}
  is_confluent A (r: Endorelation A): Proposition :=
    (star r).is_diamond
\end{alba}




\subsection{Draft}



\begin{alba}
   stripe_lemma
     A
     (r: Endorelation A)
     (rdia: r.is_diamond)
     : all a b c: r a b => plus r a c => some d: plus r b d and r c d :=
       f a b c rab pac :=
         inspect pac case
           init a c :=
             via some d: r b d and r c d

           step a x c :=
             {: goal a   ->   b
                     |        |
                     v        v
                     x   ->   e?
                     +        +
                     v        v
                     c  ->    d?  :}
             via some e: r x e and r b e         -- 'r' is diamond
             via some d: plus r e d and r c d   -- via 'f x e c'
             where
                _: plus r b d := step b e d
                _: r c d := _
\end{alba}


\begin{alba}
  transitive_closure_of_diamond_is_diamond
    A
    (r: Endorelation A)
    (rdia: r.is_diamond)
    : (plus r).is_diamond :=
      f a b c pab pac: some d: plus r b d and plus r c d :=
        inspect pab case
          init a b :=
            {: goal  a -> b
                     +    +
                     v    v
                     c -> d?  :}
            stripe_lemma r a b c

          step a x b :=
            {: goal a -> x  +> b
                    +    +     +
                    v    v     v
                    c -> e? +> d? :}
            via some e: plus r x e and r c e       -- via stripe_lemma r a x c
            via some d: plus r x e and plus r b d  -- via 'f x b e'
            where
               _: plus r c d := step c e d

\end{alba}











\newpage
\section{Lists}
\label{sec:certprog-lists}


In the following we use the definition of the list type from the prelude.

\begin{alba}
  class
    List (A:Any): Any
  create
    [] : List A
    (^): A -> List A -> List A
\end{alba}
%
Note that the operator $\caret$ is right associative and has the
highest precedence of all arithmetic operators.



\subsection{Concatenation and Reversal}


A standard definition of list concatenation looks like

\begin{alba}
  (+) A (a b: List A): List A :=
    inspect a case
      [] :=
        b
      h ^ t :=
        h ^ (t + b)
\end{alba}
Note that the operator $+$ is left associative.


List concatenation is associative
%
\begin{alba}
  sum_associates A (a b c: List A): a + b + c = a + (b + c) :=
    inspect a case
      h ^ t :=
        goal where
          goal: h ^ t + b + c = h ^ t + (b + c) :=
            via
               h ^ t + b + c
               h ^ (t + b + c)      -- def (+)
               h ^ (t + (b + c))    -- hypo
               h ^ t + (b + c)      -- def (+)
          hypo: t + b + c = t + (b + c) :=
            sum_associates t b c
\end{alba}


\begin{alba}
  nil_right_neutral A (a: List A): a + [] = a :=
    inspect a case
      h ^ t :=
        goal where
          goal: h ^ t + [] = h ^ t :=
            via
              h ^ t + []
              h ^ (t + [])   -- def (+)
              h ^t           -- hypo
          hypo: t + [] = t :=
            nil_right_neutral t
\end{alba}

\vskip 10mm
\noindent STATUS: VERY DRAFT


\begin{alba}
  class
    List (A:Any): Any
  create
    []: List A
    (::): A -> List A -> List A

  class
    is_sum A: List A -> List A -> List A -> Proposition
      -- s.is_sum a b means 's' ist the concatenation of 'a' and 'b'.
  create
    init a:
      a.is_sum [] a
    step e s a b:
      s.is_sum a b
      =>
      (e :: s).is_sum (e :: a) b
\end{alba}




\begin{alba}
  sum_nil (s a: List A) (p: s.is_sum [] a): s = a :=
    inspect
      p
    case
      init a :=
        _    -- s = a in that case
      step e xy x y: (e :: xy).is_sum (e :: x) y :=
          -- [] = e :: x is a contradiction
        _
\end{alba}



\begin{alba}
  sum_unique (a b s1: List A) (p: s1.is_sum a b)
    : all s2: s2.is_sum a b => s1 = s2 :=
    inspect
      p
    case
      init b :=
        --  a = []
      step ???? :=
        ????
\end{alba}

\begin{alba}
  sum_associates
   (a b c ab bc ab_c a_bc: List A)
   : ab.is_sum a b
     => bc.is_sum
     => ab_c.is_sum ab c
     => a_bc.is_sum a bc
     => ab_c = a_bc  :=
     inspect
       a
     case
       [] :=
         ???
       h :: t :=
         ???
\end{alba}




\begin{alba}
  class
    is_reverse A: Endorelation (List A)
      -- a.is_reverse b means 'a' is 'b' reversed
  create
    init:
      [].is_reverse []
    step ra a s e:
      ra.is_reverse a
      => s.is_sum ra [e]
      => s.is_reverse e :: a
\end{alba}

\begin{alba}
  reverse_prepend A (a b: List A): List A :=
      -- prepend the reversal of 'a' in front of 'b'
    inspect
      a
    case
      [] :=
        b
      hd :: tl :=
        tl.reverse_prepend e :: b
\end{alba}


%%% Local Variables:
%%% mode: latex
%%% TeX-master: "main_alba_design"
%%% End:
