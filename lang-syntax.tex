\section{Syntax}

This chapter would be better named "Experimental Language Concepts".





\subsection{Recursion Schemes}

Goal: Separation of recursion and algebraic operations on subterms.

\subsubsection{Example: Expression Evaluation}

\begin{alba}
    class Ring :=
        zero:   Ring
        one:    Ring
        add:    Ring -> Ring -> Ring
        times:  Ring -> Ring -> Ring
        negate: Ring -> Ring

    eval: Ring -> Int :=
            -- specific evaluation function
        case
            zero :=
                0
            one  :=
                1
            add a b :=
                eval a + eval b
            times a b :=
                eval a * eval b
            negate a :=
                - eval a
\end{alba}



\noindent Parametrized expression with F-Algebra

\begin{alba}
    Algebra (F: Any -> Any) (A: Any): Any :=
        F A -> A

    class RingF  (A: Any): Any :=
        zero:   RingF A
        one:    RingF A
        add:    A -> A -> RingF A
        times:  A -> A -> RingF A
        negate: A -> RingF A

    evalF: Algebra RingF Int :=
        -- note: Algebra RingF Int = (RingF Int -> Int)
        case
            zero :=
                0
            one :=
                1
            add a b :=
                a + b
            times a b :=
                a * b
            negate a :=
                - a
\end{alba}

\noindent Generic evaluation function using F-Algebra

\begin{alba}
    eval (A: Any) (eF: Algebra RingF A): Expression -> A
    :=
        case
            zero :=
                eF zero
            one :=
                eF one
            add a b :=
                eF <| add (eval a) (eval b)
            times a b :=
                eF <| times (eval a) (eval b)
            negate a :=
                eF <| negate (eval a)
\end{alba}



\subsubsection{Example: List}

\begin{alba}
    class List A :=
        []: List A
        (+:): A -> List A -> List A

    class ListF A B :=
        nil := ListF A B
        cons: A -> B -> ListF A B

    eval (A B: Any) (e: ListF A B -> B): List A -> B
    :=
        case
            [] :=
                e nil
            head +: tail :=
                eval tail |> cons head |> e
\end{alba}




\subsubsection{Top Down Traversal}

The previous generic evaluation function used a postorder traversel of the tree.
It first evaluated the children, and then evaluated the node. This traversal is
essentially bottoms up. For lists this recursion corresponds to
\code{foldRight}.

There is also a top down traversal e.g. for lists \code{foldLeft}.

\begin{alba}
    foldLeft (A B: Any) (f: A -> B -> B) (start: B): List A -> B
    :=
        case
            [] :=
                start
            head +: tail :=
                foldLeft f (f head start) tail
\end{alba}
%
This recursion has the advantage that it is tail recursive. How can we encode
this function as a generic recursion with some algebra?

The following function does the job.

\begin{alba}
    evalRight (A B: Any) (e: Algebra (ListF A) B): List A -> B
    :=
        f (e nil) where
        f start :=
            case
                [] :=
                    start
                head +: tail :=
                    f (e <| cons head start) tail
\end{alba}

Does this work in the generic case? Probably not, because it violates the
algebra.





\subsection{Anamorphism}

Example: List unfold

\begin{alba}
    -- Note: Function might not terminate!! Not valid in Alba!!
    unfold (A B: Any) (f: A -> Maybe (B, A)) (a: A): List B :=
        inspect f a case
            nothing :=
                []
            just (b, a) :=
                b +: unfold f a

    -- example: Range function
    range (a n: Natural): List Natural :=
            -- start from 'a', 'n' elements
        unfold f (a,n)
        where
            f :=
                case
                    (a,n) :=
                        inspect n case
                            0 :=
                                nothing
                            k.successor :=
                                just (a, (a + 1, k))
\end{alba}





\begin{alba}
    factorial (n: Natural): Natural :=
        foldRight (\i acc := acc * i) 1 (range 1 n)

        -- foldLeft would be more efficient (tail recursive)
\end{alba}


\begin{alba}
    -- Note: Function might not terminate!! Not valid in Alba!!
    ana (A B: Any) (make: B -> ListF A B): B -> List A :=
        f where
            f b1 :=
                inspect make b1 case
                    nil :=
                        []
                    cons a b2 :=
                        a +: f b2
\end{alba}







\begin{alba}
\end{alba}








\subsection{Abstract Datatypes: Different Approach}



The 3 most important abstract data types.

\begin{alba}
    class Decidable (A: Any): Any :=
        decidable: (all (a b: A): Decision (a = b) (a /= b)) -> Decidable A

    class Sortable (A: Any): Any :=
        sortable
            (r: Endorelation A)
            :
                r.isTransitive
                ->
                (all a b: Comparison r a b)
                ->
                Sortable A

    class SortableKey (A: Any): Any :=
        sortable
            (r: Endorelation A)
            :
                r.isTransitive
                ->
                r.isAntisymmetric
                ->
                (all a b: Comparison r a b)
                ->
                Sortable A

    -- with the following comparison type:
    class Comparison (A: Any) (r: Endorelation A) (a b: A): Any :=
        lessThan: r a b -> not r b a -> Comparison r a b

        equivalent: r a b -> r b a -> Comparison r a b

        greaterThan: not r a b -> r b a -> Comparison r a b
\end{alba}
%
Every sortable type has a transitive endorelation and a comparison function. The
presence of a comparison function guarantees reflexivity. Reason:
An object of type \code{Comparison a a} can only be constructed by
\code{equivalent} which guarantees \code{r a a}. The other constructors imply a
contradiction.

Furthermore the endorelation of a sortable type is linear. Reason: The
comparison function does not allow that two elements are unrelated i.e.
\code{not r a b and not r b a} is not possible by construction.

Every sortable key has in addition to a sortable an antisymmetric order
relation. The antisymmetry guarantees that {\em equivalence} implies {\em
equality}.

Note that a sortable key is decidable. Reason: Equivalence proves equality. Less
than and greater than together with equality lead to a contradiction, therefore
both prove inequality.




\begin{alba}
\end{alba}







We define an inductive type which captures the essence of a partial order.

\begin{alba}
    class
        PartialOrder (A: Any): Any
    :=
        partialOrder
            (r: Endorelation A)
            :   r.isReflexive -> r.isAntisymmetric -> r.isTransitive
                -> PartialOrder A
\end{alba}


We introduce a section to define more functions on partial orders.

\begin{alba}
    section
        A: Any
        (<=): Endorelation A
        reflexive: (<=).isReflexive
        antisymmetric: (<=).isAntisymmetric
        transitive: (<=).isTransitive
    :=
        lowerBound (p: Predicate A) (x: A): Proposition :=
            all y: p y => x <= y

        lowerBoundTransitive
            (x y: A) (p: Predicate A)
            : x <= y => lowerBound p y => lowerBound p x
        :=
            \ xLEy lbY z pZ: x <= z :=
                transitive xLEy <| lbY pZ

        least (p: Predicate A) (x: A): Proposition :=
            p x and lowerBound p x

        upperBound (p: Predicate A) (x: A): Proposition :=
            all y: p y => y <= x

        greatest (p: Predicate A) (x: A): Proposition :=
            p x and upperBound p x

        infimum (p: Predicate A)  (x: A): Proposition :=
            greatest (lowerBound p) x

        supremum (p: Predicate A)  (x: A): Proposition :=
            least (upperBound p) x

        ...
\end{alba}








\begin{alba}
\end{alba}





\begin{alba}
    class Group A :=
        group
            (zero: A)
            (plus: A -> A -> A)
            (invers: A -> A)
            (assoc: all a b c: (a.plus b).plus c = a.plus(b.plus c))
            (rightNeutral: all a: a.plus zero = a)
            (rightInvers:  all a: a.plus a.invers = zero) = e)

    -- Having this, the following shall be provable
    section
        (A: Any)
        (g: Group A)
    :=
        leftNeutral: all a: g.zero.(g.plus) a = a := ...

        leftInvers:  all a: a.(g.invers).(g.plus) a = g.zero
\end{alba}

Some syntactic sugar might be necessary to make it more readable.



\begin{alba}
    intGroup: Group Int :=
        group 0 (+) (-) assocProof rightNeutralProof rightInversProof
\end{alba}

This approach has the advantage, that no name clashes are possible, inheritance
is trivial etc. (like modules in Coq).



\begin{alba}
\end{alba}





\subsection{Indentation Blocks}

There are certain constructs which require that the following construct must
be indented. E.g. a class definition consists of a declaration and a
definition.

{\small
\begin{verbatim}
   -- ( construct ) means that construct must be indented after
   -- each newline

   type_definition ::=
     'class' (class_declaration) class_definition

   class_definiton ::=
     'create' ({constructor})     -- {a}: zero or more a
\end{verbatim}
}


\begin{alba}
  class Color create red green blue

  class Natural create 0 succ(Natural)

  class Natural create
    0
    succ(Natural)

  class
    Natural: Any
  create
    0
    succ(Natural)
\end{alba}


Big advantage of indentation blocks: There is no need to put any form of
parentheses around the blocks (like \code{begin ... end}), the programmer
indicates by indentation which construct must be considered as part of the
block or to an outer block.


\begin{alba}
  class
    (<=): Natural -> Natural -> Prop
  create
    zero (a:Natural): 0 <= a
    succ (a,b:Natural): a <= b  =>  a.succ <= b.succ
\end{alba}



\begin{alba}
  -- Abstract classes
  abstract class Partial_order section (PO)  -- implicit type 'PARTIAL_ORDER'
    (<=) (a,b:PO): Propostion
    reflexive: all(a:PO)
      a <= a
    antisymmetric: all(a,b:PO)
      a <= b  and  b <= a
      =>  a = b
    transitive: all(a,b,c:PO)
      a <= b and b <= c
      => a <= c
\end{alba}
%
or more verbose
\begin{alba}
  abstract class
    Partial_order
  section(PO:PARTIAL_ORDER)
    (<=) (a,b:PO): Propostion
    reflexive: all(a:PO)
      a <= a
    ...
\end{alba}

A function definition
\begin{alba}
  (+) (a,b:Natural): Natural :=
    inspect a case
      0 := b
      n.succ :=
        (n + b).succ

  -- or more verbose
  (+) (a,b:Natural): Natural :=
    inspect
      a
    case
      0 := b
      n.succ :=
        (n + b).succ
\end{alba}


\paragraph{Requirements for the Parser}
The start of an indentation block must be marked. Indentation blocks are
nested. At the beginning of a block the parser does not yet know how much
indentation the programmer chooses. This is evident after the first newline
with an indentation more than the indentation of the outer block.

The first actual indentation within an indentation block fixes the indentation
of the block.

All opened indentation blocks form some kind of tabulators. We remain within
the inner indentation block as long the indentation is at least the
indentation of the block. As soon as the next indentation is less than the
minimum indentation, the indentation block is closed.


\paragraph{Open Questions}
\begin{itemize}
\item Is more indentation allowed? For the innermost block maybe yes. But
  closing a block has to land exactly on one of the indentation levels of the
  outer blocks which are determined by the first indented line of the
  block. Otherwise the indentation might look very flattery.

\item
  However it might be better to enforce strict indentation. With strict
  indentation the nesting is explicit. I.e. either put an indented block
  completely on a line or indent exactly the same amount after each newline
  where the indentation of the first indented line determines the needed
  indentation of the block.
\end{itemize}













\subsection{Parentheses}

There is the functional programming style like
%
\begin{alba}
  f (x:X) (y:Y) (z:Z) ... : RT

  f x y z ...
\end{alba}
%
and the usual style
%
\begin{alba}
  f (x:X, y:Y, z:Z, ...): RT

  f (x, y, z, ...)
\end{alba}
%
%
Dot notation
%
\begin{alba}
  set.has(e)                           set `has` e
                                       e `elem` set

  x.f(a,b,c)                           (x `f` a) b c
                                       -- note
                                       x `f` a b c =  x `f` (a b c)
\end{alba}
%
Note that
%
\begin{alba}
  f (x:X, y:Y, z:Z, ...): RT
\end{alba}
%
could be used in functional style as well. Furthermore it could be possible to
use single quotes to convert identifiers with more than one character into
binary operators like
%
\begin{alba}
  e 'element_of' set

  set 'has' e
\end{alba}
%

Let's look at some more advanced examples:
%
\begin{alba}
  (>>=) ( A B: Any, r1: Spec(A), r2: A -> Spec(B),
          m: IO(A, r1),       -- r1 inferable from this type!
          f: all ( a )
               IO(B, r2(a))   -- r2 inferable from this type!
        )
        : IO( B,
              (s1, b, s3) :=
                  some (a,s2)
                    r1(s1,a,s2) and
                    r2(a,s2,b,s3))

  -- versus

  (>>=) ( A B: Any, r1: Spec A, r2: A -> Spec B,
          m: IO A r1,         -- r1 inferable from this type!
          f: all a:
               IO B (r2 a)    -- r2 inferable from this type!
        )
        : IO  B
              (\ s1 b s3 :=
                  some a s2:
                    r1 s1 a s2 and
                    r2 a s2 b s3)

\end{alba}
%
Since the right hand side of \code{:=} is greedy, parentheses can be avoided
around the function definition.
%
\begin{alba}
  -- function definition with backslash lambda

          IO  B
              \ s1 b s3 :=
                  some (a, s2)
                    r1 s1 a s2 and
                    r2 a s2 b s3
\end{alba}
%
does not give us a lot, because with types it looks like
%
\begin{alba}
  \ (a:A, b:B, ...) :=
        ...

  -- versus

  (a:A, b:B, ...) :=
        ...
\end{alba}


\begin{alba}
  open (name: String):
       IO (Maybe Descriptor)
          (s1, fd, s2) :=
             inspect fd case
                just (fd) :=
                   s1.open_files `not_has` fd
                   and
                   s2 = s1[open_files := s1.open_files + fd]
                nothing :=
                   s2 = s1

  putc (c:Char, fd:Descriptor)
       : IO Unit
            (s1, _, s2) :=
                s1.open_files `has` fd and
                s1 = s2
\end{alba}


\begin{alba}
  x.f.f.f                  set.has x

  -- instead of

  f (f (f x))              has set x
                           set `has` x
  -- is still possible
\end{alba}




\subsection{Parentheses 2}

\begin{alba}
  -- function definition
  f (a: A, b: B, ... ): RT := ...
      -- ^ comma necessary to terminate type

  -- without arg types
  f a b ...: RT := ...

  -- without types
  f a b ... := ...

  -- anonymous
  \(a: A, b:B, ...): RT := ...

  \ a b ... := ...
\end{alba}


\begin{alba}
  -- dependent types
  all (a:A, b:B, ...): RT

  -- without arg types
  all a b ...: RT

  -- combined
  f: all(a:A, b:B, ...): RT :=  -- ':=' binds lower than quantifier
    \ a b ... := ...
\end{alba}



\subsection{Operators}


\subsubsection{General Remarks}

Binary operators have precedence and associativity.

Unary operators can be viewed as binary operators where the left operand is
missing. E.g. logical negation $\lnot\, a$ can be viewd as $\cdot\, \lnot\, a$
where $\cdot$ marks the invisible left operand.

We use juxtaposition for function application i.e. $f \, a$ is the
application of the function $f$ to the argument $a$. We view the whitespace
between the function and the argument as an invisible binary operator i.e. we
can treat $f \, a$ as $f \, \odot\, a$ where $\odot$ is the invisible operator
with left associativity.



Precedence and associativity rules can save us parentheses as shown in the
following examples where $+,\times$ are left associative, $\triangleleft$ is
right associative, function application is left associative and the unary
operator $\lnot$ is treated as a binary operator with right associativity.
%
$$
\begin{array}{lll}
  a + b + c
  &  \to
  &   (a + b) + c

  \\

  a \triangleleft b \triangleleft c
  &  \to
  & a \triangleleft (b \triangleleft c)

  \\

  a \times b + c
  &  \to
  &   (a \times b) + c

  \\

  a + b \times c
  &  \to
  &   a + (b \times c)

  \\

  f \odot a \odot b & \to & (f \odot a) \odot b

  \\

  \lnot \, \lnot \, a & \to &  \lnot \, (\lnot \, a)

  \\

  \cdot\, \lnot\, \cdot\, \lnot\, a
  & \to
  &  \cdot\, \lnot\, (\cdot\, \lnot\, a)
\end{array}
$$


We group the operators by precedence levels.

Different operators having the same precedence are treated as the same
operators. E.g. $+$ and $-$ are addition operators. They have the same
precedence (lower that multiplication operators, higher than logical operator)
and are left associative.


\subsubsection{Relational Operators}

Relational operators don't have associativity. E.g. having the relational
expression
$$
a \le b < c
$$
%
it does not make sense to parse it as $(a \le b) < c$ or $a \le (b <
c)$. There are 2 possibilities
%
\begin{enumerate}

\item Treat expressions like $a \le b < c$ as syntactically wrong

\item Parse $a \le b < c$ as $a \le b \,\land\, b < c$.

\end{enumerate}
%
The first one is simpler to implement, the second one is more elegant and
closer to usual conventions in mathematics.




\subsubsection{Unary Operators}

The arithmetic negation $-$ is the only operator which can be prefix (unary)
or infix (binary).
$$
%
\begin{array}{lllll}

  a + b - c
  & \to
  & (a + b) - c

  \\

  - a + b
  & \to
  & (-a) + b
  & \quad\mid
  & (\cdot - a) + b

  \\

  a + -b
  & \to
  & a + (-b)

  \\

  - a \times b
  & \to
  & (-a) \times b

  \\

  a \times -b
  & \to
  & a \times (-b)

  \\

  - a^b
  & \to
  & -( a^b)
  & \quad\mid
  & \cdot - (a ^ b)

  \\

  \lnot\, a \le b
  & \to
  & \lnot\, (a \le b)
\end{array}
$$
%
Unary minus has a higher precedence than addition and multiplication, but
lower precedence than exponentiation. Binary minus is an addition
operator and has the same precedence and associativity as $+$.




\subsubsection{Pretty Printing}

Once the syntax tree is parsed, the parentheses have been stripped off and the
expression structure is encoded in the syntax tree.

If we want to print an expression we have to insert the necessary parentheses.
The precedence and associativity of the operators guide the insertion.

We assign to each subexpression an optional precedence/assoc information.

If the information is missing then the subexpression is treated as atomic i.e.
without parentheses.

If the upper node in the syntax tree has higher precedence, then parentheses
have to be inserted. When the upper node has lower precedence, then no
parentheses are needed.

If the upper node and the subexpression have the same precedence (i.e.
equivalent operators with same associativity), then the associativity and the
location of the subexpression as a left or right operand decides the need for
parentheses.

Parentheses are needed if the subsexpression is the left operand and we have
right associativity or the subsexpression is the right operand and we have
left associativity or if we have non-associativity.

However for non-associativity it might be elegant to print $a \le b < c$
instead of the more complex $a \le b \,\land\, b < c$. This is not trivial
because the case $a \le b \,\land\, b < c$ has to be distinguished from the
case $a \le b \,\land\, c < d$.








%%% Local Variables:
%%% mode: latex
%%% TeX-master: "main_alba_design"
%%% End:
