\section{Syntax}

\subsection{Indentation Blocks}

There are certain constructs which require that the following construct must
be indented. E.g. a class definition consists of a declaration and a
definition.

{\small
\begin{verbatim}
   -- ( construct ) means that construct must be indented after
   -- each newline

   type_definition ::=
     'class' (class_declaration) class_definition

   class_definiton ::=
     'create' ({constructor})     -- {a}: zero or more a
\end{verbatim}
}


\begin{alba}
  class Color create red green blue

  class Natural create 0 succ(Natural)

  class Natural create
    0
    succ(Natural)

  class
    Natural: Any
  create
    0
    succ(Natural)
\end{alba}


Big advantage of indentation blocks: There is no need to put any form of
parentheses around the blocks (like \code{begin ... end}), the programmer
indicates by indentation which construct must be considered as part of the
block or to an outer block.


\begin{alba}
  class
    (<=): Natural -> Natural -> Prop
  create
    zero (a:Natural): 0 <= a
    succ (a,b:Natural): a <= b  =>  a.succ <= b.succ
\end{alba}



\begin{alba}
  -- Abstract classes
  abstract class Partial_order section (PO)  -- implicit type 'PARTIAL_ORDER'
    (<=) (a,b:PO): Propostion
    reflexive: all(a:PO)
      a <= a
    antisymmetric: all(a,b:PO)
      a <= b  and  b <= a
      =>  a = b
    transitive: all(a,b,c:PO)
      a <= b and b <= c
      => a <= c
\end{alba}
%
or more verbose
\begin{alba}
  abstract class
    Partial_order
  section(PO:PARTIAL_ORDER)
    (<=) (a,b:PO): Propostion
    reflexive: all(a:PO)
      a <= a
    ...
\end{alba}

A function definition
\begin{alba}
  (+) (a,b:Natural): Natural :=
    inspect a case
      0 := b
      n.succ :=
        (n + b).succ

  -- or more verbose
  (+) (a,b:Natural): Natural :=
    inspect
      a
    case
      0 := b
      n.succ :=
        (n + b).succ
\end{alba}


\paragraph{Requirements for the Parser}
The start of an indentation block must be marked. Indentation blocks are
nested. At the beginning of a block the parser does not yet know how much
indentation the programmer chooses. This is evident after the first newline
with an indentation more than the indentation of the outer block.

The first actual indentation within an indentation block fixes the indentation
of the block.

All opened indentation blocks form some kind of tabulators. We remain within
the inner indentation block as long the indentation is at least the
indentation of the block. As soon as the next indentation is less than the
minimum indentation, the indentation block is closed.


\paragraph{Open Questions}
\begin{itemize}
\item Is more indentation allowed? For the innermost block maybe yes. But
  closing a block has to land exactly on one of the indentation levels of the
  outer blocks which are determined by the first indented line of the
  block. Otherwise the indentation might look very flattery.

\item 
  However it might be better to enforce strict indentation. With strict
  indentation the nesting is explicit. I.e. either put an indented block
  completely on a line or indent exactly the same amount after each newline
  where the indentation of the first indented line determines the needed
  indentation of the block.
\end{itemize}

%%% Local Variables:
%%% mode: latex
%%% TeX-master: "main_alba_design"
%%% End:
