\section{Type System}

Types live in \emph{universes}. Initially there are the three universes
\code{Proposition}, \code{Any} and \code{Box} ($\Prop$, $\Any$ and $\Box$ in
mathematical formulas).

Universes are called \emph{sorts} as well.

A new type definition has to indicate in which universe the type lives. The
default universe is $\Any$.

\begin{alba}
  class List(A) create [] (^)(A,List(A))

  -- is equivalent to
  class List(A:Any): Any create ...

  -- i.e. the identifier 'List' has the type (or better kind)
  Any -> Any
\end{alba}

A proposition has be be defined explicitly to live in the universe $\Prop$.
%
\begin{alba}
  class (<=): Natural -> Natural -> Proposition create
    zero (a:Natural): 0 <= a
    next (a,b:Natural): a <= b => a.succ <= b.succ

  -- (and): Proposition -> Proposition -> Proposition
  class (and) (a,b:Proposition): Proposition create
    conj: a => b => a and b
\end{alba}

Universes are ordered. The order has to be wellfounded. The top element is
$\Box$. This is the biggest universe. Initially only $\Prop < \Any < \Box$ is
valid.

New universes are introduced by defining abstract classes.
\begin{alba}
  abstract class Partial_order
    section (PO:Partial_order)
      (<=) (a,b:PO): Proposition
      reflexive: all(a:PO)
        a <= a
      antisymmetric: all(a,b:PO)
        a <= b  and  b <= a
        =>  a = b
      transitive: all(a,b,c:PO)
        a <= b and b <= c
        => a <= c
\end{alba}
%
By default for all new abstract classes $A$ the relation $A < \Any$ is
valid.

Since the class \code{Natural} has a relation \code{<=} which is reflexive,
antisymmetric and transitive, we can state that \code{Natural} lives not only
in the biggest universe \code{Any} but also in the more restricted universe
\code{Partial\_order}. This is indicated by the declaration of an
inheritance relation
%
\begin{alba}
  class Natural inherit Partial_order
\end{alba}

Next we define a decidable partial order
%
\begin{alba}
  abstract class Decidable_partial_order: Any
    section (DPO: Decidable_partial_order)
      (<=) (a,b:DPO): Proposition
      (<=) (a,b:DPO): Decision (a <= b)

      reflexive: all(a:PO)
        a <= a
      ...
\end{alba}
%
Initially the universe of a decidable partial order is minimal and is a
subuniverse of $\Any$. But it is evident that the decidable partial order is a
subuniverse of a partial order and we can state this fact by
%
\begin{alba}
  abstract class Decidable_partial_order inherit Partial_order
\end{alba}



\begin{alba}
  abstract class
    Functor: Any -> Any
  section (A, B: Any, F: Functor)
    map: (A->B) -> F(A) -> F(B)
    -- with operator []
    []:  (A->B) -> F(A) -> F(B)
\end{alba}


\begin{alba}
  abstract class Monad: Any -> Any
    section (A, B: Any, M:Monad)
      return: A -> M(A)
      (>>=): M(A) -> (A -> M(B)) -> M(B)

      [] (f:A->B, m:M(A)): M(B) :=
        m >>= (a) := return @ f(a)

      <*> (mf:M(A->B), m:M(A)): M(B) :=
        mf >>= (f) := f[m]

      -- in do notation
      [] (f:A->B, m:M(A)): M(B) :=
        do
          a := m
          return @ f(a)

      <*> (mf:M(A->B), m:M(A)): M(B) :=
        do
          f := mf
          f[m]
\end{alba}








%%% Local Variables:
%%% mode: latex
%%% TeX-master: "main_alba_design"
%%% End:
