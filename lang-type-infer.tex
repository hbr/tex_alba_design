\section{Type Inference}

A global (constant or function) is uniquely found by its name and its
signature. Many globals can share the same name as long as their signature is
different.

Looking up the name results in a list of globals. If the list is empty, then a
global with that name does not exist.

Locals shadow all previously declared names. I.e. either one local or a list
of globals is returned by looking up a name.

In an expression a name is used with a certain number of essential arguments
(i.e. with an arity at its call site). This has to be compared with the actual
arity. The actual arity must be sufficient for the required arity at the call
site.

If the arity is sufficient, the formal arguments can be pushed into the
context and a type remains as the result type which must be unified with the
required result type (possibly instantiating substitutable type variables).


\paragraph{Loop up a global} It is a function call with $n$ essential
arguments. The $n$ arguments and the corresponding type variables and proof
arguments can be extracted and pushed into the \emph{global} context with
dummy variables. If the global is not unique we get more than one context. The
context might have different inferable variables and different types for the
dummy variables.

\paragraph{Required types}
If it is explicitly given (result type of a function in a function definition
or explicitly typed expression), then it is only one type and not a set of
types.

If the expression is used as an argument of a function then there can be
multiple required types. Anyhow we have a context (or a set of contexts) and
for each context an explicit type or a pointer to an argument (whose type is
the required type).



%%% Local Variables:
%%% mode: latex
%%% TeX-master: "main_alba_design"
%%% End:
