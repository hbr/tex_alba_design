\section{Expressions}



\subsection{Primitive Expressions}
%-----------------------------------------

\begin{enumerate}

\item Sorts: \code{Any}, \code{Proposition}

\item Literal Values

    \begin{enumerate}

    \item Numbers: 1, 2, ... , 3456, 2.0, ...

    \item Strings: "Hello"

    \item Characters: 'A'

    \end{enumerate}

\item Identifiers and Operators:
    \code{isEmpty}, \code{(+)}, \code{(,)} \code{[]},
    \code{Int}, \code{List}

\end{enumerate}


All primtive expressions have an evident type. The type of numbers can be
ambiguous. E.g. \code{1:Int} and \code{1:Float} are valid typing judgements. For
characters and strings the types \code{Character} and \code{String} exist.


The sorts have the types \code{Proposition:Any}, \code{Any:Any(1)}. The type
\code{Any(1)} is not expressible in the source language, but it exists under the
hood as an infinite type hierarchy \code{Any(0)}, \code{Any(1)}, \code{Any(1)},
\code{Any(2)}, ... where \code{Any} is the same as \code{Any(0)}.

Binary operators like \code{+} and \code{-} must be enclosed in parentheses in
order to form a primitive expression. Operator expressions see below.








\subsection{Function Application}
%-----------------------------------------

A function application is a function expression applied to one or more
arguments. Function application is done by juxtaposition. In principle there are
only functions with one argument. However a function can return another function
which can be applied to further arguments. This is the way to make functions
accepting more than one argument.

\begin{alba}
    f a b c ...

    -- parsed as
    ( ... ((f a) b) c) ...  -- i.e. left associative
\end{alba}
i.e. \code{f a} represents a function which can be applied to another argument
and \code{(f a) b)} is a function which can be applied to another argument.

Alba is indentation sensitive. If a function application spans more than one
line, then the arguments appearing on subsequent  lines have to be indented. The
indentation must be aligned with the position of the first argument
\begin{alba}
    -- legal                                illegal
    f a b c d                               f  a b
      e f                                    c

    f                                       f  a b
        a                                        c
        b
        c                                   f
                                            a b c
\end{alba}




Any expression in a function position must have a type like
\begin{alba}
    f: all (x: A): B

    -- or if 'x' does not occur in 'B'
    f: A -> B
\end{alba}

A function application can have implicit arguments. I.e. if you write in the
source code the term \code{f a b c} the compiler has to find out the implicit
arguments. I.e. it has to find the positions and the values of the implicit
arguments.
\begin{alba}
    f a b c

    f . a . . b c    -- '.' is an implicit argument inferred by the compiler
\end{alba}

As an example lets look at the operator \code{|>} which serves to write function
application as an operator expression.
\begin{alba}
    f a

    a |> f
\end{alba}
The \code{|>} operator has the following type
\begin{alba}
    (|>): all (A: Any) (a: A) (B: Any): (A -> B) -> B
\end{alba}
%
The following example shows how the compiler must infer the implicit arguments
%
\begin{alba}
    (|>) 1: all (B: Any): (Int -> B) -> B

    -- implicit made explicit
    (|>) Int 1

    -- one more explicit argument
    (|>) 1 (+): Int -> Int
    -- implicit made explicit
    (|>) Int 1 Int (+)

    -- one more explicit argument
    (|>) 1 (+) 2: Int
    -- implicit made explict
    (|>) Int 1 Int (+) 2
\end{alba}

The operator \code{|>} has two explicit arguments and two implicit arguments.
Note that even if \code{|>} has only two explicit arguments, the expression
\code{(|>) 1 (+) 3} i.e. the expression \code{(|>)} applied to three arguments
is legal. This is because the implicit argument \code{B} is instantiated with
the type of \code{(+)} which is \code{Int -> Int -> Int}.






\vskip 5mm
\subsection{Typed Expressions}
%-----------------------------------------

Every expression can get a type annotation.
\begin{alba}
    exp: Type

    -- precedences and associativity:
    -- expression                       parsed as
    1 + 3 - 4: Int                      (1 + 3 - 4): Int
    \ x := exp: E                       \ x := (exp: E)
    1: Int: Any                         1: (Int: Any)
    prf: a < b: Proposition             prf: (a < b: Proposition)
\end{alba}
%
Note that types are expressions as well. Therefore you can regard \code{:} as a
binary operator with relatively low precedence and right associativity. However
it is not possible to define functions with the name "\code{:}".






\vskip 5mm
\subsection{Operator Expressions}
%-----------------------------------------

Operator expressions of the form
\begin{alba}
    a + b * c / d^4

    -- parsed as
    a + ((b * c) / (d^4))
\end{alba}
are possible in Alba with the usual precedence and associativity rules.

Besides the usual operators, the comma operator is possible in Alba. It has
relatively low precedence and right associativity. It is mainly used for tuples.

%
\begin{alba}
                                            -- parsed as
    (Int, String)
    (1, "hello")
    (Int, String, Character)                (Int, (String, Character))
    (\x := "hello" + x, 2)                  ((\x := "hello" + x), 2)
    (all x y: x = y, Int)                   ((all x y: x = y), Int)
\end{alba}

\noindent Function application binds stronger than a binary operation.
\begin{alba}
                                    -- parsed as
    f a + g b                       (f a) + (g b)
\end{alba}








\vskip 5mm
\subsection{Anonymous Functions}
%-----------------------------------------

General form:
\begin{alba}
    \ x y ... := exp                    -- untyped

    \ (x: A) (y: B) ... : R := exp      -- fully typed

    \ x y ...
    :=
        exp                         -- defining expression must be indented
\end{alba}
The type annotations are optional for argument types and the result type.
However they might be needed sometimes to make a function expression
unambiguous.

\vskip 3mm
\noindent Examples:
\begin{alba}
    \ x := x + 1

    \ x y := x + y

    \ (x: Int) (y: Int): Int    -- fully typed
    :=
        x + y                   -- defining expression must be indented
\end{alba}



\subsubsection{Type Inference}

In a function expression the types of the arguments and the result type are
optional. This makes the code more readable in many cases. However this means
that the compiler must infer the types, because the intermediate language is
fully typed and each local variable must get a type.

\vskip 3mm
\noindent Possibilities for the compiler to infer the types:
\begin{itemize}

\item There is an expected type for the function expression.

\item The function expression is applied to arguments and the types of the
arguments can be inferred.

\item The defining expression \code{exp} {\em uses} some formal arguments as
actual arguments in some function calls and the signature of the called function
specifies the type of the argument.

\item The defining expression has a type and that type must be the result type
of the function expression.

\end{itemize}

\noindent In the following situations the compiler cannot infer the needed types:
\begin{alba}
    (\x y := x + y) = (\x y := x + y)
    -- 'x' and 'y' can be either strings or numbers

    (\x y := x + y) a b
    -- where 'a' and 'b' are untyped local variables as well
\end{alba}






\vskip 5mm
\subsection{Local Definitions}
%-----------------------------------------


Local definitions allow to insert variables into an expressions and define them
in a \code{where} block. The definitions can be constants or functions.
%
%
Examples:
\begin{alba}
    1 + a where a := 100

    1 + f 10 where f x := 2 * x ^ 2 + 7 * x + 10

    1 + f 10 where
        f x := 2 * x            -- local definition must be indented

    1 + f 10 + g 20 where       -- multiple definitions must be aligned
        f x := 2 * x
        g x := x - 10

    1 + f 10 20 where           -- multiple definitions can build on eachother
        f x y := 2 * x + g y
        g x := 1 + x

    1 + f 30 where
        f (x: Int): Int         -- fully typed
        :=
            1 + x               -- defining expression must be indented
\end{alba}

The definition dependency is inside out. The outer definitions must not depend
on inner definitions.

Local definitions are just a method to write more expressive code. For each
where expression there is a functionally equivalent expression using anonymous
functions.

\begin{alba}
                                        -- equivalent
    1 + a where a := 20                 (\a := 1 + a) 10

    1 + f 10 where f x := 2 * x         (\f := 1 + f 10) (\x := 2 * x)

    1 + f 10 20 where                   (\g :=
        f x y := 2 * x + g y                (\f := 1 + f 10 20)
        g x := 1 + x                            (\x := 2 * x + g y)
                                        ) (\x := 1 + x)
\end{alba}
%
Note how the usage of local definitions improves readability.










\vskip 5mm
\subsection{Function Types}
%-----------------------------------------


General form:
%
\begin{alba}
    all (x: A) (y: B) ... : R

    A -> B -> ... -> R      -- if variables x, y, ... do not occur in types

    all a b: a => b => a    -- untyped
    -- where
    (=>): Proposition -> Proposition -> Proposition
\end{alba}

Untyped variables can be used only if the untyped variables appear in subsequent
argument types or the result type. Otherwise the compiler has no chance to infer
the argument types of the untyped variables. But even if untyped variables
appear in subsequent types, the compiler might not be able to infer the types.

\begin{alba}
    all x y: x = y              -- cannot determine the type of 'x'
    -- where
    (=): all (A: Any): A -> A -> Proposition

    all A: A -> A
    -- can be
    all (A: Any): A -> A
    all (A: Sortable): A -> A
    ...
\end{alba}
%
In the first example the compiler can only infer that both arguments must have
the same type, but it cannot find the specific type. In the second case the
variable \code{A} could be \code{Any} or any abstract type.


\subsubsection{Implicit Arguments}

MISSING!!!









\vskip 5mm
\subsection{List Expressions}
%-----------------------------------------

General form:
%
\begin{alba}
    [a, b, c, ... ]
\end{alba}

All elements of a list must have the same type. The compiler tries to transform
such a literal list into an expression of the form
%
\begin{alba}
    a +: b +: c ... +: []
\end{alba}
%
i.e. it tries to find an operator pair (\code{[]}, \code{(+:)}) with the
following signatures
%
\begin{alba}
    []:   all (A: Any): F A
    (+:): all (A: Any): A -> F A -> F A
    -- where
    F: Any -> Any
\end{alba}
%
In the standard library there are the types \code{List} and \code{Array} which
satisfy the requirement for \code{F}. Any container having the corresponding
operators can be filled with a list expression.






\vskip 5mm
\subsection{Pattern Match}
%-----------------------------------------

\begin{alba}
    inspect
        index0          -- indices are optional
        index1
        ...
        match-exp
    case
        {elim-function} -- elimination function is optional

        c0 (a: A) (b:B) ... : R := -- types are optional
            exp
        ...
\end{alba}

Equivalent form without \code{inspect} part.

\begin{alba}
    (case
        {elim-function} -- elimination function is optional

        c0 (a: A) (b:B) ... : R := -- types are optional
            exp
        ...
    ) index0 index1 ... match-exp
\end{alba}

Internal form:

$$
\case(\lambda \ybold^\Abold x^T. R, [f_0, f_1, \ldots])
:
\Pi \ybold^\Abold x^T. R
$$
