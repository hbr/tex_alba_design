\section{Relation to Other Typed Lambda Calculi}
\label{sec:lang-other-typed}


In the section \emph{Intermediate
  Language}~\ref{sec:lang-intermediate-language} a typed lambda calculus has
presented which serves as an intermediate language for Alba programs. The
presented lambda calculus is fairly general. In this chapter we explain, how
other typed lambda calculi are contained as special cases.

In particular we demonstrate that all typed lambda calculi of the Barendregt
cube are contained as special cases.


\subsection{Barendregt Cube}

For all typed calculi in the Barendregt cube we have no inductive types, no
abstract types, no pattern match and no fixpoint constructs. Furthermore we
have only the sorts $\Prop$ and $\Any$. I.e. our terms
are defined by the grammar
$$
\begin{array}{lllll}
  t &:= & \Prop \mid \Any  & \text{sorts}
  \\
    & \mid & x & \text{variable}
  \\
    & \mid & f \, a & \text{application}
  \\
    & \mid & \lambda x^A. e & \text{abstraction}
  \\
    & \mid & \Pi x^A.B & \text{dependent product}
\end{array}
$$.

We have no subtyping relation and only one axiom
$$
 [] \vdash \Prop : \Any
$$

Although it sounds a little bit funny any type $T$ with $ T : \Prop$ is a
\emph{normal} type i.e. \emph{normal} types are propositions.

The power and expressibility of a typed lambda calculus is determined mainly
by the allowed rules to form function types i.e. dependent products. We need
the generic rule
$$
\rulev
{
  \Gamma \vdash A : s_1
  \\
  \Gamma, x^A \vdash B : s_2
}
{ \Gamma \vdash \Pi x^A . B : s_2}
$$

\begin{tabular}{| l | l | l | l |}
  \hline
  & $s_1$ & $s_2$ &
  \\
  \hline
  1. fun       & $\Prop$ & $\Prop$ & functions from objects to objects
  \\
  2. polyfun   & $\Any$  & $\Prop$ & polymorphic functions
  \\
  3. polytype  & $\Any$  & $\Any$  & polymorphic types
  \\
  4. deptype   & $\Prop$ & $\Any$  & dependent types
  \\
  \hline
\end{tabular}

In the typed calculus of the intermediate language the typing rule for
products is a littly bit more restrictive, but by combining the product rule
and the subtype rule with the sorts $\Prop$, $\Any$ and $\Box$ the same effect
as the above product rules can be achieved.

Depending on which product rules we allow we get the main typed calculi of the
Barendregt cube by
\begin{enumerate}
\item Simply typed lambda calculus (also called $\lambda_\to$): rules =
  \{fun\}

\item System $F$ (also called $\lambda 2$): rules = \{fun, polyfun\}

\item System $F_\omega$ (also called $\lambda 2\omega$): rules = \{fun,
  polyfun, polytype\}

\item Calculus of Constructions (also called $\lambda C$): rules = \{fun,
  polyfun, polytype, deptype\}
\end{enumerate}

These are the most interesting calculi of the Barendregt cube. In total there
are $2^3 = 8$ calculi for all other possible combinations of the rules.





\subsection{Church Encoding of Inductive Types}

Since we don't have inductive types, we need some other mechanism to encode
nontrivial functions. In order to represent interesting features we need at
least system $F$ or sometimes system $F_\omega$. The expressive power of
simply typed lambda calculus is rather limited.

We have other two features which are very powerful
\begin{enumerate}
\item Impredicativity: Propositions (aka normal types) are impredicate
  i.e. any term whose type is a proposition can via polymorphic functions
  receive any type (i.e. its own type as well) as an argument.

\item Church Encoding: It is possible to find Church Encodings for inductive
  types which allows to write all recursive functions which can be written in
  the inductive type.
\end{enumerate}

In order to find a Church encoding for values of inductive type we let us
guide by a folding function which can be defined for any inductive type.



\subsubsection{Booleans}

With inductive types we define booleans as
\begin{alba}
  class Boolean: Any create
    true:  Boolean
    false: Boolean
\end{alba}
%
with the folding function
%
\begin{alba}
  fold (b:Boolan) (A:Any) (x y:A): A :=
    inspect b case
      true :=
         x
      false :=
         y
\end{alba}


Since \code{Boolean} has two constructors we encode boolean values as
functions with three arguments. The first argument is the result type and the
other two arguments represent the alternatives.

$$
\begin{array}{lllll}
  B & :=
  & \Pi A^\Prop . A \to A \to A
  &:
  & \Prop
  \\
  \text{true} &:=
  & \lambda A x y . x
  &: & B
  \\
  \text{false} &:=
  & \lambda A x y . y
  &: & B
\end{array}
$$

The type of a boolean value is a function which receives a type, two objects
of that type and returns an object of that type.

The function $\text{true}$ returns the first of the two objects, the function
$\text{false}$ returns the second of the two objects. A boolean value encodes
a decision between two alternatives.  $\text{true}$ decides for the first
alternative and $\text{false}$ decides for the second.



\subsubsection{Pairs}

With inductive types we define pairs as
\begin{alba}
  class (,) (A B: Any): Any create
    (,): A -> B -> (A,B)
\end{alba}
%
with the folding function
%
\begin{alba}
  fold A B (p: (A,B)) (C:Any) (f: A -> B -> C): C :=
    inspect p case
      (a,b) :=
         f a b
\end{alba}

Using the folding function we can encode the projections on the first and the
second element of the pair.
\begin{alba}
  first A B (p: (A,B)): A :=
     p.fold (\ x y := x)

  second A B (p: (A,B)): B :=
     p.fold (\ x y := y)
\end{alba}

From that we derive the Church encoding

$$
\begin{array}{lllll}
  P & :=
  & \lambda A B . \Pi C^\Prop . (A \to B \to C) \to C
  &:
  & \Prop \to \Prop \to \Prop
  \\
  \text{pair} &:=
  & \lambda A B x y C f . f x y
  &: & \Pi A^\Prop B^\Prop x^A y^B.  P \, A \, B
  \\
  \text{first} &:=
  & \lambda A B p . p A (\lambda x y. x)
  &: & \Pi A^\Prop B^\Prop. P\, A\, B \to A
  \\
  \text{second} &:=
  & \lambda A B p . p A (\lambda x y. y)
  &: & \Pi A^\Prop B^\Prop. P\, A\, B \to B
\end{array}
$$






\subsubsection{Natural Numbers}

\begin{alba}
  class Natural: Any create
    0: Natural
    succ: Natural -> Natural

  iterations (n:Natural) (A:Any) (f:A -> A) (s:A): A :=
      -- iterations is in that case a better name for the folding function
    inspect n case
      0 :=
        s
      succ k :=
        f (fold k f s)

  (+) (a b: Natural): Natural :=
    a.iterations succ b

  (*) (a b: Natural): Natural :=
    a.iterations (\ s := s + b) 0

  (^) (a b: Natural): Natural :=
    b.iterations (\p := p * a) (succ 0)
\end{alba}

\noindent The Church encodings:
$$
%
\begin{array}{lllll}
  \Natural & :=
  & \Pi A^\Prop . (A \to A)  \to A \to A
  &:
  & \Prop
  \\
  0 & :=
  & \lambda A f s . s
  &:
  & \Natural
  \\
  \text{succ} & :=
  & \lambda k A f s . f (k f s)
  &:
  & \Natural \to \Natural

  \\

  (+) & :=
  & \lambda a b. a\, \Natural\,\text{succ}\, b
  &:
  & \Natural \to \Natural \to \Natural

  \\

  (*) & :=
  & \lambda a b. a\, \Natural\, (\lambda s. s + b) 0
  &:
  & \Natural \to \Natural \to \Natural

  \\

  (\caret) & :=
  & \lambda a b. b\, \Natural\, (\lambda p. p * a) (\text{succ}\, 0)
  &:
  & \Natural \to \Natural \to \Natural
\end{array}
$$




%%% Local Variables:
%%% mode: latex
%%% TeX-master: "main_alba_design"
%%% End:
