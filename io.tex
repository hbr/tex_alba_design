\chapter{IO}

\section{Basics}

Input and output have a high cost in terms of processor cycles
{\small
\begin{verbatim}
   L1 cache               3 cycles
   L2 cache              14 cycles
   RAM                  250 cycles
   Disk          41 000 000 cycles
   Network      240 000 000 cycles
\end{verbatim}
}
%
\begin{description}

\item[Disk IO] When writing to a disk the read/write head has to be positioned
  to the correct sector. The programm is blocked until the disk is positioned
  and the data is transferred between the physical disk and the corresponding
  part of the ram. Doing this for one or only a couple of characters is highly
  inefficient.

\item[Network IO] Network cards transmit and receive data with a certain
  packet size (e.g. on ethernet 1492 bytes). The network driver does
  transmission of packages of the corresponding size. In order to send data it
  has to be ready to send. In order to receive data some packet must have been
  received.
\end{description}



Input/Output can be
\begin{enumerate}
\item Disk IO

\item Network IO

\item Console IO
\end{enumerate}





\section{IO Models}



\subsection{Monadic IO}

\noindent Monadic bind operator:
\begin{ocaml}
  val (>>=): 'a t -> ('a -> 'b t) -> 'b t
\end{ocaml}


\begin{ocaml}
  getc fd >>= function
  | None ->
    (* error handling *)
  | Some c ->
    (* success handling *)

  (* signature *)
  open_read: string -> file_descriptor option t
  getc: file_descriptor -> char option t

  (* complete signatures, threading state *)
  open_read: string -> state -> state * file_descriptor option
  getc: file_descriptor -> state -> state * char option

  (* because of *)
  type 'a t = state -> state * 'a
\end{ocaml}
%

\begin{itemize}
\item How to verify that the file descriptor is open?


\item Practically all IO operations can fail even if the programmer knows that
  they cannot fail. E.g. A file descriptor has been successfully opened. The
  first subsequent close operation must succeed.

\end{itemize}


E.g. the open-file operation could in the success case return a file
descriptor and a proof that the corresponding file descriptor is open.


\subsection{Hoare State Monad}

A computation $C(A,S)$ has a value type and a specification. The specification
is a relation between the prestate, the value and the poststate. The
computation can be executed only in a possible prestate. Once executed it
yields a value of type $A$ and ends in a state satisfying the postcondition.

\begin{alba}
  State: Any := ...

  Spec(A:Any): Any :=  State -> A -> State -> Proposition

  pre(A:Any,r:Spec(A), s1:State): Proposition :=
    some(a,s2) r(s1,a,s2)

  post(A:Any, r:Spec(A), s1:State, a:A): Proposition :=
    some(s2) r(s1,a,s2)


  Hoare(A:Any, r:Spec(A)): Any := (A, r)

  open_spec (s1:State, fd:Maybe(Descriptor), s2:State): Proposition :=
    inspect fd case
      just(fd) :=
        not s1.open_files.has(fd)
        and
        s1.open_files + fd = s2.open_files
      nothing :=
        s1 = s2
\end{alba}
Note that the precondition of \code{open\_spec} is always satisfied because
%
\begin{alba}
   pre(open_spec,s1) = some (fd,s2) open_spec(s1,fd,s2)
\end{alba}
%
which is trivial to satisfy. Just take \code{fd = nothing} and \code{s1 = s2}.

Now we declare the types of the function \code{open} and the type constructor
\code{IO} as
%
\begin{alba}
  IO(A:Any, spec:Spec(A)): Any

  open: IO(Descriptor, open_spec)
\end{alba}
%
The type of \code{return} is easy. The call \code{return a} just injects the
value $a$ into the IO-monad and leaves the state unchanged.
%
\begin{alba}
  return (A:Any, a:A)
         : IO( A,
               (s1,x,s2) :=
                   x  = a and
                   s1 = s2 )
\end{alba}
%
For the bind operation it is a little bit more difficult. We would expect
something like
%
\begin{alba}
  (>>=) ( A,B:Any,
          m:IO(A,r1),
          f:A -> IO(B,r2)
        ): IO(B,r3)
\end{alba}
%
where $r_1$ is the specification of the first monad and $r_2$ is the
specification of the monad $f(a)$ and $r_3$ is the specification of the
resulting monad \code{m >>= f}. However the specifications $r_1$ and $r_2$
must be given as arguments (but they are inferable) and the specification
$r_3$ must be computed as the sequence of both specifications. We end up with
something like

%
\begin{alba}
  (>>=) ( A,B: Any, r1: Spec(A), r2: A -> Spec(B),
          m: IO(A, r1),       -- r1 inferable from this type!
          f: all(a)
               IO(B, r2(a))   -- r2 inferable from this type!
        )
        : IO( B,
              (s1,b,s3) :=
                  some(a,s2)
                    r1(s1,a,s2) and
                    r2(a,s2,b,s3))
\end{alba}
%


\begin{alba}
  open (name: String):
       IO ( Maybe (Descriptor),
            (s1, fd, s2) :=
               inspect fd case
                  just (fd) :=
                     not s1.open_files.has(fd)
                     and
                     s2 = s1[open_files := s1.open_files + fd]
                  nothing :=
                     s2 = s1 )

  putc (c:Char, fd:Descriptor)
       : IO ( Unit,
              (s1, (), s2) :=
                  s1.open_files.has(fd) and
                  s1 = s2 )
\end{alba}


\vspace{3cm}
\noindent From article ``Hoare State Monad''

\begin{alba}
  section(A:Any)
    State: Any := ....
    Pre: Any := State -> Proposition
    Post: Any := State -> A -> State -> Proposition

    Hoare (pre:Predicate(State), post: Post): Any :=
      all(s1:{s:State | pre s}) {(a,s2): (A,State) | post(s1,a,s2)}
    ...
\end{alba}

The precondition is just a predicate over the state. The postcondition is a
relation between the prestate a value and a poststate.

\begin{verbatim}
  top (s:State): Proposition := true

  return (A:Any) (a:A): Hoare(A,top, (s1,x,s2) := s1 = s2 and x = a) :=
    (x,s) := (x,s)

  (>>=) (A,B:Any, m:Hoare(A,pre1,post1), f:all(a:A) Hoare(B,pre2(a),
\end{verbatim}

\subsection{Elm Architecture}


\begin{ocaml}
  update: 'model -> 'msg -> 'model * 'msg Cmd.t

  view: 'model -> 'msg Html.t

  subscriptions: 'model -> 'msg Subscriptions.t
\end{ocaml}


\subsection{CSP}













\section{Synchronous versus Asynchronous IO}

The usual IO mode is synchronous. The user requests to read from a disk file
to a buffer, the operating system handles the request and the user waits
until the request has been handled and continues with the result.

Only one request can be made at a time and during the processing of the
request the program has to wait. In the IO monad this can be handled e.g. by
%
\begin{ocaml}
  fill_buffer fd >>= fun r ->
  match r with
  | None ->
    (* error handling *)
  | Some _ ->
    (* success handling *)
\end{ocaml}
%
Note that the bind operator has the signature
%
\begin{ocaml}
  val (>>=): 'a t -> ('a -> 'b t) -> 'b t
\end{ocaml}


The programming model in a web browser and of \code{node\_js} is
asynchronous. For each possible request (e.g. a mouse click or the receiving
of a request on a server) the user registers an event handler.

Each time an event occurs, the runtime calls the event handler. The events
happen sequentially but in unpredictable order. Only one event handler is
called at a time. No parallel processing of events is possible in this event.

In order to fit asynchronous IO into monadic programming the monadic value
\code{'a t} must have a message type \code{'a} which is a message and each
message handler must generate a message of this type. The monadic function of
type \code{'a -> 'b t} decodes the message and handles the actual event. This
is called \emph{parallel wait}.

Such a parallel wait can be handled e.g. by the unix system calls
\code{select} and \code{poll} which can wait on reading into a buffer or
flushing of buffers on a collection of file descriptors.

A monadic value of type \code{'a t} must be generated from a function which
receives all the event handlers which can - in case of its event - create a
message of type \code{'a}. The function of type \code{'a -> 'b t} actually
handles the specific request. The event handlers used to generate the monadic
value of type  \code{'a t} are just the message creators.

In the monadic asynchronous modell the event handler are just message
transformers (they transform the content of the event into a specific message)
and the continuation function actually \emph{handles} the events.

An \emph{event loop} can be made with a continuation function of type \code{'a
  -> 'a t}.

If the message type changes then all event transformers must be replaced by
new ones generating the new message type from the event which can occur.

In Elm the update function has the signature
\begin{ocaml}
  update: 'model -> 'msg -> 'model * 'msg Cmd.t

  view: 'model -> 'msg Html.t

  subscriptions: 'model -> 'msg Subscriptions.t
\end{ocaml}




\section{Abstraction Layer for IO}

\subsection{Requirements}

\begin{enumerate}

\item Process environment, command line arguments. Exit.

\item Get current directory and change current directory.

\item Create and remove directories, get the content of directories, rename
  directories.

\item Node Paths: Build filenames with basename and file name according to the
  operating system (i.e. backslash or slash as separator, drive name, etc.)

\item Stat files and directories, compare modification timestamps (module
  Stats).

\item Open files for reading (existent), writing (create if not existent,
  overwrite if existent). Rename files. Remove files (unlink)

\item Read and write files with buffers (fill, flush), read and write from io
  buffers.

\item Run external programs (like curl, unzip and wget) and collect their
  results with timemout.
\end{enumerate}


\subsection{Implementation}


\section{HTTP}

HTTP works over TCP connections. A HTTP request opens a TCP connection to the
server, sends the HTTP request and receives the HTTP response.

Since HTTP/1.1 there is a keep alive mechanism which allows to use the same
TCP connection for several HTTP requests.

HTTP/1.1 introduced chunked transfer encoding to allow content on persistent
connections to be streamed rather than buffered.

HTTP pipelining is a technique in which multiple HTTP requests are sent on a
single TCP connection without waiting for the corresponding responses.

Byte serving is the process of sending only a portion of an HTTP/1.1 message
from a server to a client. Byte serving begins when an HTTP server advertises
its willingness to serve partial requests using the Accept-Ranges response
header.

\noindent Request message
\begin{enumerate}
\item Request line consisting of method url version (e.g., GET
  /images/logo.png HTTP/1.1, which requests a resource called /images/logo.png
  from the server.)

\item request header fields (e.g., Accept-Language: en)

\item an empty line (lines must be separated by <CR><LF>)

\item an optional message body
\end{enumerate}
%
Example
{\small
\begin{verbatim}
  GET / HTTP/1.1
  Host: www.example.com
\end{verbatim}
}


\noindent Respone message
\begin{enumerate}
\item a status line which includes the status code and reason message (e.g.,
  HTTP/1.1 200 OK, which indicates that the client's request succeeded.)

\item response header fields (e.g., Content-Type: text/html)

\item an empty line (lines must be separated by <CR><LF>)

\item an optional message body
\end{enumerate}
%
Example
{\small
\begin{verbatim}
  HTTP/1.1 200 OK
  Date: Mon, 23 May 2005 22:38:34 GMT
  Content-Type: text/html; charset=UTF-8
  Content-Length: 138
  Last-Modified: Wed, 08 Jan 2003 23:11:55 GMT
  Server: Apache/1.3.3.7 (Unix) (Red-Hat/Linux)
  ETag: "3f80f-1b6-3e1cb03b"
  Accept-Ranges: bytes
  Connection: close

  <html>
    <head>
      <title>An Example Page</title>
    </head>
    <body>
      <p>Hello World, this is a very simple HTML document.</p>
    </body>
  </html>
  \end{verbatim}
}


\section{HTTPS}

HTTPS is HTTP over TLS or SSL. Both TLS and SSL need cryptographic algorithms
for key exchange (Diffie-Hellman). Both might be difficult to get right.

For native ocaml we can use Cohttp to make Http request or a calls to
\code{curl}.

For ocaml over node we can use node directly, because node has a http library
to make Http requests.


\section{Open Questions}

\begin{enumerate}
\item How to turn echoing on and off in order to read passwords from standard
  input?

  The usual way to do this (for any language) is using ANSI escape
  sequences. Code 8 sets the terminal to conceal characters and code 0 resets
  the attributes, making them visible again.

  For example, in a unix shell you can test this with echo and read:
  \begin{verbatim}
  echo -e '\x1b[8m'; read varname; echo -e '\x1b[0m'
  \end{verbatim}
  To do the same from OCaml, you can output \verb|"\x1b[8m"|, read the
  password, and then output \verb|"\x1b[0m"| afterwards to switch printing
  back on.

\end{enumerate}
%%% Local Variables:
%%% mode: latex
%%% TeX-master: "main_alba_design"
%%% End:
