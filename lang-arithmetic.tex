\section{Arbitrary Precision Arithmetic}

We have a base $B$ and digits $0 \le d < B$ i.e. the greatest digit is $B-1$.
\begin{itemize}

\item The digit sequence $(d_0 d_1 d_2 \ldots d_{n-1})_B$ is interpreted as the number
  $(\ldots((d_0 B + d_1)B + d_2) + \ldots) + d_{n-1}$ i.e.

    $$ d_0 B^{n-1} + d_1 B^{n-2} + \ldots + d_{n-2} B + d_{n-1} $$


\item The result of adding two digits is at most $2B-2$.

\item The result of adding two digits and a carry ($0$ or $1$) is at most
  $2B-1$.

\item For a two digit sequence where the first digit is $1$ we get
  $(1 d)_B = 1 B + d \le 2 B - 1$

\item For the four digits $a,b,c,d$ we get $a b + c + d < B^2$ because of $a b
  \le (B - 1)^2 = B^2 - 1 - (2B-2)$ and $c + d \le 2B-2$. Furthermore $(a b +
  c + d)/B < B$ is valid.
 \end{itemize}


Primitive operations
\begin{enumerate}
\item Addition and substraction of one-place integers, giving a one-place
  answer and a carry.

\item Multiplication of a one-place integer with another giving a two-place
  answer.

\item Division of a two-place integer by a one-place integer, provided that
  the quotient is a one-place integer, and yielding also a one-place
  remainder.

\end{enumerate}






\subsection{Division}



\subsubsection{Basics}

In order to define the division algorithm we have to define integer division and
prove some properties of integer division.

\begin{definition}

    We write the integer division of the nonegative number $a$
    divided by the positive number {b} as

    $$\intdiv{a}{b}$$

    \noindent $\intdiv{a}{b}$ is the number $n$ satisfying the specification

    $$
    nb \le a \le (n+1) b - 1
    $$

    Note that the specification defines the number $n$ uniquely.
\end{definition}

\begin{theorem}\label{intdiv-theorem}
    Any number $n$ satisfying the specification
    %
    $$n b \le a \le (n+1) b - 1$$
    %
    for $0 \le a$ and $0 < b$ satisfies the condition
    %
    $$ n b f \le a f + i \le (n+1) b f - 1 $$
    %
    for all $0 \le i < f$. I.e.
    %
    $$
    \intdiv{a}{b} = \intdiv{af}{bf} = \intdiv{af + i}{bf}
    $$
    is valid.

    \begin{proof}
        We prove this claim by the following chain of order preserving
        transformations.
        %
        $$
        \begin{array}{lllll}
            n b &\le& a &\le& (n+1) b - 1
            \\
            nbf &\le& af &\le& (n+1) b f - f
            \\
            nbf + i &\le& af + i &\le& (n+1) b f - (f - i)
            \\
            nbf     &\le& af + i &\le& (n+1) b f - 1
        \end{array}
        $$
        %
        The last step is valid because of $0 \le i < f$. The first condition is
        equivalent to
        $$ n = \intdiv{a}{b}$$
        The last condition is equivalent to
        $$ n = \intdiv{af + i}{bf}.$$
    \end{proof}
\end{theorem}


\subsubsection{Basic Division}

In this subsection let $u = (u_0 u_1 \ldots u_n)_B$ and $v = (v_1 v_2 \ldots
v_n)_B$ and $u / v < B$.
%
We want to find $q := \floor{u / v}$.

We start with the initial guess
$$
\hat{q} := \text{min}( \intdiv{u_0 B + u_1}{v_1}, B - 1)
$$


\begin{theorem}
    The initial guess is an upper bound for the correct value i.e. $q \le
    \hat{q}$.

    \begin{proof}
        Since $q < B$ because of $u/v < B$ the claim is evident for $\hat{q} = B
        - 1$. Therefore we treat the case $\hat{q} < B - 1$.

        We always have
        $$
            (u_2 \ldots u_n)_B < B^{n-1}
        $$
        by definition of positional numbers. If we use the theorem about integer
        divisions~\ref{intdiv-theorem} with $f := B^{n-1}$ and $i := (u_2 \ldots
        u_n)_B$ we get
        $$
        \begin{array}{lll}
            \hat{q}
            &:=&
            \intdiv{u_0 B + u_1}{v_1}
            \\
            \\
            &=&
            \intdiv{(u_0 B + u_1) B^{n-1}}{v_1 B^{n-1}}
            \\
            \\
            &=&
            \intdiv{(u_0 B + u_1) B^{n-1} + (u_2 \ldots u_n)_B}{v_1 B^{n-1}}
            \\
            \\
            &=&
            \intdiv{u}{v_1 B^{n-1}}
        \end{array}
        $$

        In general we have $\floor (u / w_1) \le \floor(u / w_2)$ if $w_2 \le
        w_1$ because dividing by a bigger value leads to a smaller result. Since
        $v_1 B^{n-1} \le v$ we get
        $$
            \intdiv{u}{v} \le \intdiv{u}{v_1 B^{n-1}}
        $$
        which proves the claim $q \le \hat{q}$.
    \end{proof}
\end{theorem}
