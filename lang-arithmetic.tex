\section{Arbitrary Precision Arithmetic}

We have a base $B$ and digits $0 \le d < B$ i.e. the greatest digit is $B-1$.
\begin{itemize}

\item The digit sequence $(d_0d_1d_2 \ldots d_{n-1})_B$ is interpreted as the number
  $(\ldots((d_0 B + d_1)B + d_2) + \ldots) + d_{n-1} $

\item The result of adding two digits is at most $2B-2$.

\item The result of adding two digits and a carry ($0$ or $1$) is at most
  $2B-1$.

\item For a two digit sequence where the first digit is $1$ we get
  $(1 d)_B = 1 B + d \le 2 B - 1$

\item For the four digits $a,b,c,d$ we get $a b + c + d < B^2$ because of $a b
  \le (B - 1)^2 = B^2 - 1 - (2B-2)$ and $c + d \le 2B-2$. Furthermore $(a b +
  c + d)/B < B$ is valid.
 \end{itemize}


Primitive operations
\begin{enumerate}
\item Addition and substraction of one-place integers, giving a one-place
  answer and a carry.

\item Multiplication of a one-place integer with another giving a two-place
  answer.

\item Division of a two-place integer by a one-place integer, provided that
  the quotient is a one-place integer, and yielding also a one-place
  remainder.

\end{enumerate}





%%% Local Variables:
%%% mode: latex
%%% TeX-master: "main_alba_design"
%%% End:
