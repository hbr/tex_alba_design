\chapter{Javascript}


\section{Js\_of\_ocaml}



\subsection{Base Types}

{\small
\begin{verbatim}
     int        int                     Number
     float      float                   Number
     bool       bool Js.t               Boolean
     string     Js.js_string Js.t       String
     array      Js.js_array Js.t        Array
\end{verbatim}
}


\subsection{Typing Javascript Objects}

Javascript objects are given types of the shape
%
\begin{ocaml}
  <m_1 : t_1; ...; m_n : t_n> Js.t
\end{ocaml}
%
using a phantom object type. The methods m\_i stands for the field of the
Javascript object. For instance, a Javascript object of type:

\begin{ocaml}
   < data : js_string t Js.prop;
     appendData : js_string t -> unit Js.meth> Js.t
\end{ocaml}
%
has a property data containing a Javascript string, and a method appendData
taking a Javascript string as argument and returning no value.


\subsection{OCaml and Javascript functions}

OCaml and Javascript do not follow the same calling convention. In OCaml,
functions can be partially applied, returning a function closure. In
Javascript, when only some of the parameters are passed, the others are set to
the \code{undefined} value. As a consequence, it is not possible to call a
Javascript function from OCaml as if it was an OCaml function, and conversely.

Calling Javascript functions: At the moment, there is no syntactic sugar for
calling Javascript functions. You should use either \code{Js.Unsafe.call} or
\code{Js.Unsafe.fun\_call}, depending whether you want \code{this} to be bound
to some particular object in the function body or not.


\subsection{Tail Calls}

Tail recursive functions are optimized into a loop.

Mutually tail recursive functions are optimized into a trampoline where the
trampoline is not called at each recursive calls but e.g. at each 50 recursive
call.









\section{Node}


\subsection{Needed Node Commands}

\subsubsection{\code{process.argv}}

The process.argv property returns an array containing the command line
arguments passed when the Node.js process was launched. The first element will
be process.execPath. See process.argv0 if access to the original value of
argv[0] is needed. The second element will be the path to the JavaScript file
being executed. The remaining elements will be any additional command line
arguments.



\subsubsection{\code{process.exit([code])}}

Exit the process with an exit code (default: 0).

\subsubsection{\code{process.env}}

The environment object of the current process. Example:
\begin{js}
  {
    TERM: 'xterm-256color',
    SHELL: '/usr/local/bin/bash',
    USER: 'maciej',
    PATH: '~/.bin/:/usr/bin:/bin:/usr/sbin:/sbin:/usr/local/bin',
    PWD: '/Users/maciej',
    EDITOR: 'vim',
    SHLVL: '1',
    HOME: '/Users/maciej',
    LOGNAME: 'maciej',
    _: '/usr/local/bin/node'
  }
\end{js}

\subsubsection{\code{process.cwd()}}

Get the current working directory.

\subsubsection{\code{process.chdir(directory)}}

The process.chdir() method changes the current working directory of the
Node.js process or throws an exception if doing so fails (for instance, if the
specified directory does not exist).

Note that this feature does not have a callback. It is called synchronuously.

\subsubsection{\code{fs.mkdir(path,callback)}}

We do not need options.

\begin{js}
  fs.mkdir('/bla', (err) => {console.log(err);});

  // return
  { [Error: EACCES: permission denied, mkdir '/bla']
    errno: -13,
    code: 'EACCES',
    syscall: 'mkdir',
    path: '/bla' }
\end{js}


\begin{js}
  fs.mkdir('./_build', (err) => {console.log(err);});

  // return
  { [Error: EEXIST: file already exists, mkdir './_build']
    errno: -17,
    code: 'EEXIST',
    syscall: 'mkdir',
    path: './_build' }
\end{js}

\begin{js}
  fs.mkdir('./tmp', (err) => {console.log(err);});

  // return
  null
\end{js}




\subsubsection{\code{fs.rmdir(path, callback)}}

Asynchronous rmdir(2). No arguments other than a possible exception are given
to the completion callback.

Using fs.rmdir() on a file (not a directory) results in an ENOENT error on
Windows and an ENOTDIR error on POSIX.



\subsubsection{\code{fs.readdir(path[, options], callback)}}

Callback \code{(err,files) => {...}}

Reads the contents of a directory. The callback gets two arguments (err,
files) where files is an array of the names of the files in the directory
excluding '.' and '..'.

\subsubsection{\code{fs.mkdtemp(prefix,callback)}}

Creates a unique temporary directory. Callback \code{(err,folder) =>
  {... }}.

Generates six random characters to be appended behind a required prefix to
create a unique temporary directory.




\subsubsection{\code{fs.unlink(path, callback)}}

Asynchronously removes a file or symbolic link. No arguments other than a
possible exception are given to the completion callback.

fs.unlink() will not work on a directory, empty or otherwise. To remove a
directory, use fs.rmdir().




\subsubsection{\code{fs.stat}}

\begin{js}
  // The commands
  const fs = require('fs');
  fs.stat(".travis.yml", (err,stats) => {console.log(err);
                                         console.log(stats)});
  // return on success
  null
  Stats {
    dev: 16777220,
    mode: 33188,
    nlink: 1,
    uid: 501,
    gid: 20,
    rdev: 0,
    blksize: 4096,
    ino: 8618330134,
    size: 416,
    blocks: 8,
    atimeMs: 1556653904784.4294,
    mtimeMs: 1556653904730.4136,
    ctimeMs: 1556653904730.4136,
    birthtimeMs: 1556653904729.9878,
    atime: 2019-04-30T19:51:44.784Z,
    mtime: 2019-04-30T19:51:44.730Z,
    ctime: 2019-04-30T19:51:44.730Z,
    birthtime: 2019-04-30T19:51:44.730Z }

  // return on failure
  { [Error: ENOENT: no such file or directory, stat '.travis.yaml']
    errno: -2,
    code: 'ENOENT',
    syscall: 'stat',
    path: '.travis.yaml' }
\end{js}

Note: To check the existence of a file or a directory better open it and treat
the error case.

\subsubsection{\code{fs.open(path, flags[, mode], callback)}}

Needed flags: Read: \code{'r'}, Write: \code{'w'} (File is created if it does
not exist).

The flag \code{'r'} returns an error if the file does not exist.

Note: On windows opening a file with \code{'r'} fails with \code{EPERM}. The
\code{'r+'} flag is needed in that case. However files name \code{.alba} do
not automatically have the hidden attribute and should be readable, writeable
without any problem.

Mode: default \code{0o666} i.e. readable and writable. Mode is only effective
if the file is created.

\subsubsection{\code{fs.close(fd, callback)}}

Callback \code{(err) => {...}}.






\subsection{Streams}

Readable streams generate their own buffers. If data is availabe on a readable
stream the even \code{readable} is emitted. An \code{error} event can be
emitted at any time e.g. when the stream cannot be created (file cannot be
opened for a file stream).

\code{end} signals the end of the stream. \code{close} signals the closure of
the file descriptor. The \code{end} and \code{close} events are emitted by
\code{fs.createReadStream(..)} at the end of the stream.

Writable streams can be buffered infinitely. Therefore either wait for the
callback on the \code{write} method (\code{ws.write(buf,cb)}) or stop writing
if \code{write} returns \code{false} and wait for the \code{drain} event.

A stream has to be represented internally as a mutable structure which has to
store the occurrence of the events \code{error,end}.

The method \code{fs.createReadStream(..)} fires an \code{open} event, when the
file has been opened successfully. We have to wait for this event (or
\code{error}) after opening a readable stream on a file.

Note: Event handlers are unregistered by \code{rs.off('end', handler)}.

\begin{js}
  require('http').createServer( function(req, res) {
    var rs = fs.createReadStream('/path/to/big/file');
    rs.on('data', function(data) {
      if (!res.write(data)) { // data buffered internally
        rs.pause();}
    });
    res.on('drain', function() {
      rs.resume();
    });
    rs.on('end', function() {
      res.end();
    });
  }).listen(8080);
\end{js}

If a call to \code{write} returns false, the \code{drain} event will be
emitted when it is appropriate to resume writing data to the stream.


Another example:
\begin{js}
  // Write the data to the supplied writable stream one million times.
  // Be attentive to back-pressure.
  function writeOneMillionTimes(writer, data, encoding, callback) {
    let i = 1000000;
    write();
    function write() {
      let ok = true;
      do {
        i--;
        if (i === 0) {
          // last time!
          writer.write(data, encoding, callback);
        } else {
          // see if we should continue, or wait
          // don't pass the callback, because we're not done yet.
          ok = writer.write(data, encoding);
        }
      } while (i > 0 && ok);
      if (i > 0) {
        // had to stop early!
        // write some more once it drains
        writer.once('drain', write);
      }
    }
  }
\end{js}












\section{This keyword}


In the global execution context (outside of any function), this refers to the
global object whether in strict mode or not.

A simple call of a function is e.g. \code{f(a1,a2,...)}. But the function can
be called with an object and this is bound to this object during the call
(\code{f.apply(obj,a1,a2,...)} or \code{f.call(obj,a1,a2,...)}).


In arrow functions, this retains the value of the enclosing lexical context's
this. In global code, it will be set to the global object.







\section{Integer Arithmetic}


Javascript can only safely represent integers $i$ in the range
$$ -2^{53} < i < 2^{53}.$$

\begin{js}
  Number.MAX_SAFE_INTEGER = Math.pow(2, 53)-1; // 9_007_199_254_740_991
  Number.MIN_SAFE_INTEGER = -Number.MAX_SAFE_INTEGER;

  BigInt(Number.MAX_SAFE_INTEGER) + 2n;
  // -> 9_007_199_254_740_993n
\end{js}


Signed 32 bit integer arithmetic:
\begin{js}
  function inc(x){
    x = x|0;  // x:int
    return (x + 1)|0;
  }

  0xffffffff|0   // -1
  0xffffffff     // 4_294_967_295 = 2^32 - 1

  -1 >>> 0         // 0xffffffff
  0xffffffff >> 0  // -1
\end{js}


\code{a >> 0} converts to Int32 (like \code{a|0}, \code{a >>> 0} converts to
UInt32.

\begin{js}
  var product = Math.imul(a, b); // 32 bit modulo arithmetic

  Math.imul(2, 4);          // 8
  Math.imul(-1, 8);         // -8
  Math.imul(-2, -2);        // 4
  Math.imul(0xffffffff, 5); // -5
  Math.imul(0xfffffffe, 5); // -10
\end{js}



\section{Compilation to Javascript}

Datatypes:

\begin{enumerate}
\item Booleans: Values \code{true} and \code{false}
\item Numbers: IEEE floating point numbers
\item Strings
\item Objects: Maps from strings to objects
\item Arrays
\end{enumerate}

Javascript functions have a fixed arity.





%%% Local Variables:
%%% mode: latex
%%% TeX-master: "main_alba_design"
%%% End:
