\section{Read Eval Print Loop (repl)}

You can type in an expression and the repl parses the expression, builds a
well typed term (if possible) and evaluates and prints it (if possible).

Not all terms can be evaluated and printed. A function has usually no print
representation (is not an instance of \code{show} in Haskell speak).

Values which have a canonical print representation:
\begin{enumerate}

\item Numbers
\item Characters
\item Strings
\item Arrays if the content type is printable.
\item Algebraic types if all contained values are printable.
\end{enumerate}


Builtin functions are functions and therefore do not have a print
representation. However the result of builtin functions is printable if the
function is applied to a complete set of arguments. E.g. $1 + 6$ can be
printed, but $(2+)$ not (because it is the function $\lambda x. 2 + x$).

$$
%
\begin{array}{lll}
  3 + 4 &=&  (\lambda x y. x + y)\, 3 \,4
  \\
        &=& (\lambda y. 3 + y) \,4
  \\
        &=& 3 + 4
\end{array}
$$



%%% Local Variables:
%%% mode: latex
%%% TeX-master: "main_alba_design"
%%% End:
