\chapter{Basics}


\section{Terminology}

\begin{description}
\item[Module] A module is the smallest compilation unit. It consists of an
  implementation file \code{<module-name>.al} and an interface file
  \code{<module-name>.ali}. The module name must be a valid lower case Alba
  identifier (i.e. no hyphens etc.). In the use block of a module each used
  module is either a module of the same package (single name) or a module of a
  used package (name \code{p.q.m}). I.e. the module name is always the last
  component of the fully qualified name. A module name must be unique within a
  package.

\item[Package] A package is a collection of modules. It is located in a
  directory having a file named \code{alba-package.yml} describing the
  package (package name, source directory, etc.). A package name must have the
  form of a dot separated list of identifiers \code{p.q.r...} where each
  individual identifier is a valid lowercase identifier.

  A package can be published in the Alba registry. A published package named
  \code{user.repo} must be located in a github repository \code{repo} of the
  github user or organisation \code{user}. This puts a requirement on valid
  github user/organisation and repository names. If you have a github username
  which is not a valid Alba lowercase identifier, then just create an
  organisation with a valid name and put your published packages there.

\item[Workspace] A workspace is a directory containing a file named
  \code{alba-workspace.yml} (empty file, may contain later some toplevel
  information) and containing a collection of packages as direct or indirect
  subdirectories. Workspaces must not be nested. With the availablility of the
  file \code{alba-workspace.yml} the user flags that this directory starts an
  Alba workspace.

  The command \code{alba compile} issued within a workspace compiles
  everything in the workspace which needs compilation.

  The compiler creates and manages a directory named \code{.alba-build} to
  store all compiled modules of the workspace.

\item[Context] The compiler creates and manages a directory called
  \code{\$HOME/.alba} where it stores all installed packages and some global
  configuration data.

\end{description}


\section{Directories}

{\small
\begin{verbatim}
   $HOME
     - ...
       - ...
          - alba-workspace.yml
             - ...
               - pkg1
                 - alba-package.yml
                 - src
                    m1.al m1.ali ....
             - pkg2
               - alba-package.yml
               - src
                   ...
          - .alba-build
            - pkg1
              - alba-package.yml
              - m1.al m1.al.d m1.al.o  m1.ali m1.ali.d m1.ali.o ...
            - pkg2
                ...
\end{verbatim}
}


\section{Commands}


\begin{description}

\item[\code{help}] The command \code{alba help cmd} displays help text of the
  command and all available options for the command.

\item[\code{init}]
  \begin{description}

  \item[\code{context}] Initializes the home directory \code{\$HOME} of the current
    user as a context.

  \item[\code{workspace}] The command \code{alba init workspace} initializes the
    current directory as the root of a workspace

  \end{description}

\item[\code{clean}]

  \begin{description}

  \item[\code{workspace}] The command \code{alba clean workspace} cleans the
    workspace (each next compilation command compiles from scratch).


  \item[\code{context}] The command \code{alba clean context} cleans the
    context i.e. it removes all installed libraries and cleans all the
    workspaces which use some of the installed libraries.

  \item[\code{all}] The command \code{alba clean all} cleans up all workspaces
    and the context. This is the command to issue if you think that something
    has been screwed up completely.
  \end{description}


\item[\code{show}]
  \begin{description}

  \item[\code{workspaces}] Shows information about the workspaces in the
    context.

  \item[\code{installed}] Shows information about installed libraries in the
    context.

  \item[\code{packages}] Shows information about the packages in the workspace.

  \end{description}

\item[\code{compile}] The command \code{alba compile mod1 mod2 pck1 ...}
  compiles all module and packages given as arguments and all their
  dependencies within the workspace. Compilation is needed only if there have
  been modifications since the last compilation. The compile command issued
  without arguments compiles the whole workspace.

  Options: \code{-f, --force} enforces compilation even if not needed.

\item[\code{status}] The command \code{alba status mod1 mod2 pck1 ...} checks
  the compilation status of the modules/packages of the workspace given as
  arguments. Issued without arguments the compilation status of the whole
  workspace checked.


\item[\code{install}] The command \code{alba install user.repo} installs the
  registered package \code{user.repo} from the corresponding github
  repository into the context.

  If given without a version, the latest released version is installed. The
  command \code{alba install user.repo.3.2.0} installs the specific version of
  the package.

  Different version of published packages can be installed into the context.


\item[\code{remove}]


\item[\code{publish}]

\end{description}



\section{Packages}

A package needs a configuration file name \code{alba-package.yml} in its
toplevel directory.

\subsubsection{ Example of a library package}
{\small
\begin{verbatim}
  libary:
    name:   mypackage         -- can be p.q.r.... or user.repo
    source: src               -- '.' not recommended
    export: [module1, module2, ... ]   -- exported modules
    use:
      public:  [pkg1, pkg2, ... ]    -- visible to users of the library
      private: [pkg1, pkg2, ... ]    -- only private use

    -- Information needed for published packages
    version: 1.0.1
    previous_version: 1.0.0      -- if exists, needed to check proper
                                 -- semantic versioning
\end{verbatim}
}
%
Modifications in the privately used packages do not affect users of the
library.



\subsubsection{ Example of a web application}
{\small
\begin{verbatim}
  web-application:

    name: myprog.html           -- or myprog.js
    main: mod                   -- module containing the function 'main'

    source: src
    use: [pkg1, pkg2, ... ]
\end{verbatim}
}


\subsubsection{ Example of an executable package}
{\small
\begin{verbatim}
  executable:

    name: myprog.js             -- javascript program to be called
                                -- by 'node myprog.js'

    main: mod                   -- module containing the function 'main'

    source: src
    use:
      - pkg1
      - pkg2
\end{verbatim}
}

The internal format in the file \code{alba-package.sexp}:
{\small
\begin{verbatim}
  (executable (myprog.js mod src (pkg1 pkg2)))
\end{verbatim}
}
The mandatory elements are stored in a canonical order. Only optional elements
are tagged, but also stored in a canonical order.





\section{Workspace}

A workspace has a file \code{alba-workspace.yml} and contains a set of
packages. The packages are found in subdirectories of the workspace and must
not overlap.


Therfore if the workspace directory contains also a file
\code{alba-package.yml}, then the package is the only package in the
workspace.


A workspace has a build directory \code{.alba-build}

The contents of the file \code{alba-workspace.yml} looks like:
{\small
\begin{verbatim}
   packages:
     - alba.core
     - program1
     - program2
     - lib1
\end{verbatim}
}

However the added value of the content of the file is questionable, because
\begin{itemize}

\item The packages can be easily found by exploring the workspace, which needs
  to be done anyhow as long as the workspace file does not contain locations.

\item There is no dependency between workspaces. A workspace has an internal
  consistency and depends only on the context, i.e. on the installed packages
  of the context.
\end{itemize}

Therefore it might be best to use the existence of the file as a marker and
check that there are no enclosing workspaces and no nested workspaces.

An option: Use the presence of the build directory \code{.alba-build} as a
marker for a workspace. The build directory can be created with \code{alba
  init workspace}. Disadvantage: It is not visible to the user that a
directory is a workspace.




\section{Dependency Management}


\subsection{Basics}

\begin{itemize}

\item Explore the workspace
  \begin{itemize}

  \item Find the nearest ancestor directory having a {\tt\small
      alba-workspace.yml} file.

  \item Verify that there is no upper workspace directory (workspaces cannot
    be nested).

  \item Scan all subdirectories to find packages i.e. directories with an
    {\tt\small alba-package.yml} file and find all illegal nested workspaces.
  \end{itemize}

\item Analyze the package descriptions
  \begin{itemize}
  \item Unavailable packages
  \item Circular dependencies
  \end{itemize}
\end{itemize}

\subsection{\code{.al} Modification}

Recompile the source file and check it against its interface specification.

A modification of a source file does not affect other modules.


\subsection{\code{.ali} Modification}

A modification of an interface file might affect all users. A modification
only affect external users if the module is exported.

Furthermore it can invalidate an implementation file. The implementation file
might still be correct but does not satisfy its specification.

It is possible that an interface and an implementation file can be compiled
correctly standalone, but the implementation does not satisfy its
specification. We can use the filenames \code{m.al.o0} and  \code{m.al.o} to
express that the first one is correct but not yet checked against its
specification and the second to express correctness and compliance with the
specification.



\subsection{\code{alba-package.yml} Modification}


\subsubsection{Problems with internal consistency}

\begin{description}

\item[Exported modules] Only available modules can be exported. An exported
  module which is unavailable is an error in the package description.

\item[Used packages] All actually used packages within modules must be
  contained. Only packages declared in the package description as used modules
  can be used in modules of the package. Privately used packages can only be
  used privately, publicly used packages can be used privately and publicly.

  Using a package not declared in the package description as usable is an
  error in the corresponding module. Therefore a modification in the package
  description can invalidate already compiled modules.

  Furthermore it is possible that a package description uses a package which
  is not existent. This is an error in the package description.

\end{description}


\subsubsection{Problems with users}
\begin{description}

\item[Exported modules] Exporting more has no external effects, exporting less
  might invalidate other packages which use the module.

\end{description}


\subsection{Old}

{\small
\begin{verbatim}
   source               builddir

   alba-package.yml     alba-package.yml

   mod.al               mod.al       -- copy src while reading
                        mod.al.d     -- write dep while reading
                        mod.al.o     -- write obj if compiles

   mod.ali              mod.ali
                        mod.ali.d
                        mod.ali.o

   ok: mod.al(src) <= mod.al(build) <= mod.al.o
       mod.al(src) <= mod.al.d      <= mod.al.o

   mod.al(build) < mod.al(src) => modified
   mod.al.o < mod.al(build)    => cannot be compiled

   write first temporary file, then rename (i.e. mv)
\end{verbatim}
}

The files \code{alba-package.yml} and \code{mod.al.d/mod.ali.d} are needed to
generate the dependency graph. Whenever a source file has been modified, the
dependency file will be rewritten (if use block is syntactically correct) and
the source file will be copied (if syntactically ok).

The compiler only works on files if they are needed for the compilation.

First loops in the files \code{alba-package.yml} are detected and then in the
files \code{mod.al*.d}.


The status checker does not modify the build directory. It just follows the
\code{mod.al*.d} of the files to be analyzed and reports modifications,
affectations and loops.

\begin{alba}
  -- Decision: Proposition -> Proposition -> Any
  class
    Decision(a,b:Proposition): Any
  create
    left  (a)
    right (b)

  -- Decision: Proposition -> Any
  Decision (a:Proposition): Any
    := Decision(a, not a)

  -- Decider: all(A:Any) (A->Proposition) -> Any
  Decider (A:Any, p:A->Proposition): Any
    := all(x) Decision (p(x))
\end{alba}







\section{Screwed Up Workspaces or Contexts}

The Alba compiler uses hidden directories like \code{.alba},
\code{.alba-workspace} to store dependencies, workspaces, usages of installed
libraries etc. The user must not edit these directories. But the user can do
it by accident or maliciously. Therefore the Alba compiler must not rely on
correct versions of these directories.

Some error cases:
\begin{itemize}

\item The user removes a complete subtree of the filesystem. This action can
  accidently remove a package from a workspace or a workspace from an
  environment.

  Each time the compiler accesses information about a workspace registered in
  the context, it has to check if the workspace still exists and remove it
  otherwise. The same happens in case of a deleted package from a workspace. A
  deletion of a package causes all depended packages to be recompiled at the
  next issue of \code{alba compile}. This is some kind of \emph{garbage
    collection}


\end{itemize}



\section{Package Registry}

A special github repo \verb|alba.registry| is used to register published
packages. Initially only alba packages are in the registry. Later packages
from arbitrary users can be published as long as they compile successfully
with the given used packages.

The registry has a file for each package which contains a list of released
versions.

\noindent Structure of the registry
{\small
\begin{verbatim}
   packages
     alba
       core
         releases  -- text file with one line per version
       http
         releases
     user1
       pkg1
         releases
       pkg2
         releases
       ...
     ...
  .travis.yml
\end{verbatim}
}
Each file \code{releases} is a textfile which has all released version numbers
of the package in reverse chronological order with one line per version
number.

In order to release a package, the owner must have a fork of the registry,
modify the file \code{packages/user/repo/releases} (put the newest release
number as the first line of the file) and then issue a pull request of the
registry.








The alba compiler can access the registry via the github api and an
authorization token which allows to modify files in the repo representing the
registry. I.e. it must be able to create blob objects, tree objects and commit
objects. The authorization token shall not be in the alba compiler binary
as cleartext. It has to be encrypted (e.g. via MD5 hash) and decrypted before
accessing the registry.

{\bf Note:} It might not be a good idea to have the authorization token within
the alba binary. The compiler is open source, therefore anybody with enough
knowhow can get the authorization token and screw up the registry. Furthermore
personal access tokens allow access to all repos of the user which is really a
killing point.

The user must authenticate itself with its github password. The password is
not stored. But a login into the github account is done. Only authenticated
users can access the registry.

Problem: It must be made sure that only packages which compile successfully
and satisfy the semantic versioning constraints can be released.

After successful check the compiler can ask the user to authenticate and then
sign the sha1 hash of the package. Requires more thinking.

Requirements:
\begin{itemize}
\item Only successfully checked packages can be released.
\item The registry must be protected in the sense that the authorization token
  cannot be accessed illegally.
\end{itemize}




\section{Install Packages}

In order to install a registered package \code{user.repo.x.y.z} the following
steps are necessary:
\begin{enumerate}

\item Find information of the registered releases of the package
  \code{user.repo} by sending a http request to the Alba registry.

\item Check that the requested version is in the list of released packages. If
  the package has been requested without a version number then use the latest
  release of the package.

\item Get the archive (either \code{x.y.z.zip} or \code{x.y.z.tar.gz}) from
  \code{user/repo/archive} e.g. by \code{wget} and store it in the context.

\item Unpack the archive into the context (e.g. by \code{tar xzf x.y.z.tar.gz}
  or \code{unzip x.y.z.zip}.

\item Install all dependencies which are not yet installed.

\item Compile the package.
\end{enumerate}


\noindent In order to install a package we need tools to
%
\begin{itemize}
\item Make Https requests to the registry (e.g. \code{curl}).
\item Download archives (e.g. \code{wget}).
\item Unpack archives (\code{tar} or \code{unzip}).
\end{itemize}


\section{Version Numbers}

A version number has the general format

\begin{verbatim}
    major.minor.patch
\end{verbatim}

\begin{itemize}
\item increase patch level: No interface changes
\item increase minor version: Only addition of interfaces; cannot break builds
\item increase major version: Build breaking changes allowed
\end{itemize}

Avoidance of potential crashes by adding new functions: Each type is owned by
a package. A library package can only export functions if it owns at least one
type in the signature. I.e. a package cannot export functions on types which
are owned by used modules. In order to export functions with no types owned by
the package, the package has to wrap the type. E.g.

\begin{alba}
class Natural2 create
    wrap(Natural)
end

unwrap (n:Natural2): Natural :=
    inspect n case
        wrap(i) := i
    end
\end{alba}
%
The wrapping and unwrapping has no runtime cost.


\section{External Tools}

\begin{itemize}
\item wget or curl to get released github repos, .zip or .tar.gz
\item tar and/or unzip to unpack github repositories
\item curl to access the github api via http requests (only for accessing the
  registry).
\end{itemize}

Note: An archive \code{<name>.tar.gz} is just a zipped tar archive. Unzip
decompresses it and then tar can extract the files from the archive. It might
be better to use \code{<name>.zip} and decompress and extract it with one
command. The tools \code{zip} and \code{unzip} shall be available on all
platforms.



\newpage
\section{Unix vs. Windows}

\begin{itemize}
\item No interval timer for native compilation of ocaml.

  We probably don't need interval timer in the alba compiler. If needed, the
  node version of the compiler should be able to generate interval timer.

\end{itemize}







\section{Syntax and Code Layout}

\lstset{language=alba}
Declarations are separated by semicolon {\tt ';'}. In order to avoid a lot of
semicolons newlines can serve as implicit semicolons.

Newlines serve as implicit semicolons unless
\begin{itemize}
\item preceded or followed by
  \begin{itemize}
    \item an operator
    \item the keywords \lstinline!case else!
  \end{itemize}
\item preceded by
  \begin{itemize}
  \item an opening parenthesis like {\tt '(', '[‘, '\{‘ }
  \item an opening keyword \lstinline!all assert do inspect require some!
  \end{itemize}
\item followed  by
  \begin{itemize}
    \item a closing parenthesis like {\tt ')', ']‘, '\}‘}
    \item a closing keyword like \lstinline!end!
  \end{itemize}
\end{itemize}


%%% Local Variables:
%%% mode: latex
%%% TeX-master: "main_alba_design"
%%% End:
