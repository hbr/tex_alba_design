\chapter{Web Applications}
\label{chapter:web}


\section{Interoperation with Javascript}

In order to interoperate with javascript (event handling, initialization etc.)
we need decoders.

An object of type \code{Decoder Int} knows how to decode a javascript or json
object and can return an object of type \code{Result Int Error} where
\code{Error} is some decoding error object.

A decoder is like a parse combinator.


Event handler

\begin{alba}
    class
        Handler (A: Any): Any
    create
        normal: Decoder A -> Handler A
        mayStopPropagation: Decoder (A, Boolean) -> Handler A
        mayPreventDefault:  Decoder (A, Boolean) -> Handler A
        custom: ...
\end{alba}

A \code{Decoder (A, Boolean)} decodes the event and returns an object of type
\code{A} and a boolean value which indicates if propagation is stopped or
default action is prevented.

\vskip 5mm
\section{Virtual Dom}

A virtual dom is just a tree of dom nodes where the elementary nodes are text
nodes. The other nodes consist of a tag, a list of attributes and a list of
children nodes. Note that a node can consist of a tag only and an empty list of
attributes and an empty list of children.

\begin{alba}
    class
        Attribute (A: Any): Any
    create
        ???

    class
        Node (A: Any): Any
    create
        text: String -> Node A
        node: String -> List (Attribute A) -> List (Node A) -> Node A
\end{alba}


We need functions to create attributes:
\begin{alba}
    -- general
    attribute:   all (A: Any): String -> String -> Attribute A
    property:    all (A: Any): String -> JSValue -> Attribute A
            -- JSValue can be constructed with an encoder

    -- input fields
    placeholder: all (A: Any): String -> Attribute A
    value:       all (A: Any): String -> Attribute A
    type:        all (A: Any): String -> Attribute A
    onInput:     all (A: Any): (String -> A) -> Attribute A

    -- handler
    on:          all (A: Any): Decoder A -> Attribute A
    onClick:     all (A: Any): A -> Attribute A
    onCheck:     all (A: Any): (Boolean -> A) -> Attribute A -- checkbox
\end{alba}

Example
\begin{alba}
    onClick (A: Any) (a: A): Attribute A :=
        on "click" (succeed a)
    -- 'succeed a' is a trivial decoder which always returns the value
    -- 'a' without looking into the event object
\end{alba}




\vskip 5 mm
\section{Events}

When an event occurs javascript sends to all listeners of the event an event
object.

Solution in Elm:

The elm compiler converts the event object into a json string value
(\code{JSON.stringify(obj)}) and each handler has a decoder to decode the json
string into an elm message.

Disadvantage: In many cases the decoder ignores the json string and just
produces an elm message.

\section{Dom - Document Object Model}
\label{sec:dom}


\subsection{The Dom Tree}

The dom is a tree.
\begin{description}
\item[Text] The leave of the tree are text nodes.

\item[Node] The inner nodes have

  \begin{description}
  \item[Tag] Paragraph, body, div, span, ...
  \item[Attributes] A list of attributes.
  \item[Children] A list of children nodes.
  \end{description}

\end{description}

Node navigation properties (i.e accessed by \code{nd.prop}).
\begin{itemize}
\item
  parentNode
\item
  childNodes[nodenumber]
\item
  firstChild
\item
  lastChild
\item
  nextSibling
\item
  previousSibling
\end{itemize}

\begin{js}
  // child nodes
  var nodeList = elementNodeReference.childNodes;

  // a node list has a length property and element can be accessed by
  // list[i].
\end{js}

A common error in DOM processing is to expect an element node to contain text.

\begin{js}
  // append a child
  var para = document.createElement("p");
  var node = document.createTextNode("This is a new paragraph.");
  para.appendChild(node);

  // remove a child node
  var parent = document.getElementById("div1");
  var child = document.getElementById("p1");
  parent.removeChild(child);

  // replace a child node
  var parent = document.getElementById("div1");
  var child = document.getElementById("p1");
  parent.replaceChild(para, child);
\end{js}

\subsection{Attributes}


\begin{js}
  // set an attribute
  element.setAttribute(attribute, value)
\end{js}

A dom node can have a list of attributes. E.g. \code{class} or \code{style}.

An attribute is a property (set in js by \code{n.className = 'greeting'}) with
a value (in this case a string).

The \code{style} is a property whose value is an object
(\code{n.style.property = 'red'}).


\vskip 3mm
\subsection{Events}

The event handlers of an attribute are properties like \code{onclick},
\code{onmousover}, etc. Their values are the handlers which are javascript
functions \code{(event) => {...}}. The event \code{onerror} needs a handler
like \code{(event,source,lineno,colno,error) => {...}}.


When the event handler is invoked, the this keyword inside the handler is set
to the DOM element on which the handler is registered. For more details see
the this keyword documentation.

When a function is used as an event handler, its this is set to the element
the event fired from (some browsers do not follow this convention for
listeners added dynamically with methods other than addEventListener()).

\begin{js}
  n.addEventListener(event, function, useCapture)
  n.removeEventListener(event, function, useCapture)
  // optional useCapture:
  //    true - Removes the event handler from the capturing phase
  //    false- Default. Removes the event handler from the bubbling phase

  // e.g.
  n.addEventListener('click',handler);
\end{js}


\begin{verbatim}
        Capture     Bubble
           v          ^
    +------v----------^------+
    |      v   Body   ^      |
    | +----v----------^----+ |
    | |    v    Div   ^    | |
    | | +--v----------^--+ | |
    | | |     Button     | | |
    | | +----------------+ | |
    | +--------------------+ |
    +------------------------+
\end{verbatim}

Today browser use both methods, but bubble is the default.

When using on you can customize the event behavior a bit. There are two ways
to do this:

\code{ev.stopPropagation} means the event stops traveling through the DOM. So
if propagation of a click is stopped, it will not trigger any other event
listeners.

\code{ev.preventDefault} means any built-in browser behavior related to the
event is prevented. This can be handy with key presses or touch gestures.



\subsection{Repaints and Reflows}

A repaint is triggered when only visibility like background color etc. changes
but the layout is unchanged.

A reflow is needed when the layout might have changed. Adding, removing or
replacing dom nodes trigger a reflow. Reflow is costlier than repaint.

Use \code{createDocumentFragment} to assemble a new dom subtree and then add
the fragment causing only one reflow.


A common use for DocumentFragment is to create one, assemble a DOM subtree
within it, then append or insert the fragment into the DOM using Node
interface methods such as appendChild() or insertBefore(). Doing this moves
the fragment's nodes into the DOM, leaving behind an empty
DocumentFragment. Because all of the nodes are inserted into the document at
once, only one reflow and render is triggered instead of potentially one for
each node inserted if they were inserted separately.


\subsection{Event Delegation}

All events bubble by default and they have a property named \code{target}
which is the innermost node where the event occurred.

We can use e.g. the class \code{id} of the node to dispatch to the correct
event handler.



\section{Custom Elements}

{\small
\begin{verbatim}
  <my-product>
  </my-product>  <!-- hyphen '-' is mandatory
\end{verbatim}
}

\begin{js}
  class MyProduct extends HTMLElement {
    constructor () {
      super ()
      this.innerHTML = 'hello'
    }
  }

  window.customElements.define ('my-product', MyProduct)
\end{js}

%%% Local Variables:
%%% mode: latex
%%% TeX-master: "main_alba_design"
%%% End:
